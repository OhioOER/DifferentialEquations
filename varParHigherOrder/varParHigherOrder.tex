\documentclass{ximera}

%% You can put user macros here
%% However, you cannot make new environments

\listfiles

%\graphicspath{{./}{firstExample/}{secondExample/}}
\graphicspath{{./}
{aboutDiffEq/}
{applicationsLeadingToDiffEq/}
{applicationsToCurves/}
{autonomousSecondOrder/}
{basicConcepts/}
{bernoulli/}
{constCoeffHomSysI/}
{constCoeffHomSysII/}
{constCoeffHomSysIII/}
{constantCoeffWithImpulses/}
{constantCoefficientHomogeneousEquations/}
{convolution/}
{coolingActivity/}
{directionFields/}
{drainingTank/}
{epidemicActivity/}
{eulersMethod/}
{exactEquations/}
{existUniqueNonlinear/}
{frobeniusI/}
{frobeniusII/}
{frobeniusIII/}
{global.css/}
{growthDecay/}
{heatingCoolingActivity/}
{higherOrderConstCoeff/}
{homogeneousLinearEquations/}
{homogeneousLinearSys/}
{improvedEuler/}
{integratingFactors/}
{interactExperiment/}
{introToLaplace/}
{introToSystems/}
{inverseLaplace/}
{ivpLaplace/}
{laplaceTable/}
{lawOfCooling/}
{linSysOfDiffEqs/}
{linearFirstOrderDiffEq/}
{linearHigherOrder/}
{mixingProblems/}
{motionUnderCentralForce/}
{nonHomogeneousLinear/}
{nonlinearToSeparable/}
{odesInSage/}
{piecewiseContForcingFn/}
{population/}
{reductionOfOrder/}
{regularSingularPts/}
{reviewOfPowerSeries/}
{rlcCircuit/}
{rungeKutta/}
{secondLawOfMotion/}
{separableEquations/}
{seriesSolNearOrdinaryPtI/}
{seriesSolNearOrdinaryPtII/}
{simplePendulum/}
{springActivity/}
{springProblemsI/}
{springProblemsII/}
{undCoeffHigherOrderEqs/}
{undeterminedCoeff/}
{undeterminedCoeff2/}
{unitStepFunction/}
{varParHigherOrder/}
{varParamNonHomLinSys/}
{variationOfParameters/}
}


\usepackage{tikz}
\usepackage{tkz-euclide}
\usepackage{tikz-3dplot}
\usepackage{tikz-cd}
\usetikzlibrary{shapes.geometric}
\usetikzlibrary{arrows}
\usetikzlibrary{decorations.pathmorphing,patterns}
\usetikzlibrary{backgrounds} % added by Felipe
% \usetkzobj{all}   % NOT ALLOWED IN RECENT TeX's ...
\pgfplotsset{compat=1.13} % prevents compile error.

\renewcommand{\vec}[1]{\mathbf{#1}}
\newcommand{\RR}{\mathbb{R}}
\newcommand{\dfn}{\textit}
\newcommand{\dotp}{\cdot}
\newcommand{\id}{\text{id}}
\newcommand\norm[1]{\left\lVert#1\right\rVert}
\newcommand{\dst}{\displaystyle}
 
\newtheorem{general}{Generalization}
\newtheorem{initprob}{Exploration Problem}

\tikzstyle geometryDiagrams=[ultra thick,color=blue!50!black]

\usepackage{mathtools}

\title{Variation of Parameters for Higher Order Equations}%\label{Module 7-ADEF}


\begin{document}

\begin{abstract}

\end{abstract}

\maketitle

\section*{Variation of Parameters for Higher Order Equations}

\subsection*{Derivation of the method}

We assume throughout this section that the nonhomogeneous linear equation
\begin{equation} \label{eq:9.4.1}
P_0(x)y^{(n)}+P_1(x)y^{(n-1)}+\cdots+P_n(x)y=F(x)
\end{equation}
is normal on an interval $(a,b)$. We'll abbreviate this equation as
$Ly=F$, where
$$
Ly=P_0(x)y^{(n)}+P_1(x)y^{(n-1)}+\cdots+P_n(x)y.
$$
When we speak of solutions of this equation and its complementary
equation $Ly=0$, we mean solutions on $(a,b)$. We'll show how to use
the method of variation of parameters to find a particular solution of
$Ly=F$, provided that we know a fundamental set of solutions
$\{y_1,y_2,\dots,y_n\}$ of $Ly=0$.


We seek a particular solution of $Ly=F$ in the form
\begin{equation} \label{eq:9.4.2}
y_p=u_1y_1+u_2y_2+\cdots+u_ny_n
\end{equation}
where $\{y_1,y_2,\dots,y_n\}$ is a known fundamental set of solutions
of the complementary equation
$$
P_0(x)y^{(n)}+P_1(x)y^{(n-1)}+\cdots+P_n(x)y=0
$$
and $u_1, u_2, \dots, u_n$ are functions to be determined. We begin by
imposing the following $n-1$ conditions on $u_1,u_2,\dots,u_n$:
\begin{equation} \label{eq:9.4.3}
\begin{array}{rcl}
u'_1y_1+u'_2y_2+&\cdots&+u'_ny_n=0 \\
u'_1y'_1+u'_2y'_2+&\cdots&+u'_ny'_n=0 \\
&\vdots& \\
u'_1y_1^{(n-2)}+u'_2y^{(n-2)}_2+&\cdots&+u'_ny^{(n-2)}_n
=0.
\end{array}
\end{equation}
These conditions lead to simple formulas for the first $n-1$
derivatives of $y_p$:
\begin{equation} \label{eq:9.4.4}
y^{(r)}_p=u_1y^{(r)}_1+u_2y_2^{(r)}\cdots+u_ny^{(r)}_n,\ 0
\leq r \leq n-1.
\end{equation}
These formulas are easy to remember, since they look as though we
obtained them by differentiating \eqref{eq:9.4.2} $n-1$ times while
treating $u_1, u_2, \dots, u_n$ as constants. To see that
\eqref{eq:9.4.3}
implies \eqref{eq:9.4.4}, we first differentiate \eqref{eq:9.4.2} to obtain
$$
y_p'=u_1y_1'+u_2y_2'+\cdots+u_ny_n'+u_1'y_1+u_2'y_2+\cdots+u_n'y_n,
$$
which  reduces to
$$
y_p'=u_1y_1'+u_2y_2'+\cdots+u_ny_n'
$$
because of the first equation in \eqref{eq:9.4.3}. Differentiating this
yields
$$
y_p''=u_1y_1''+u_2y_2''+\cdots+u_ny_n''+u_1'y_1'+u_2'y_2'+\cdots+u_n'y_n',
$$
which reduces to
$$
y_p''=u_1y_1''+u_2y_2''+\cdots+u_ny_n''
$$
because of  the second equation in \eqref{eq:9.4.3}.
Continuing in this way yields \eqref{eq:9.4.4}.

The last equation in \eqref{eq:9.4.4} is
$$
y_p^{(n-1)}=u_1y_1^{(n-1)}+u_2y_2^{(n-1)}+\cdots+u_ny_n^{(n-1)}.
$$
Differentiating this yields
$$
y_p^{(n)}=u_1y_1^{(n)}+u_2y_2^{(n)}+\cdots+u_ny_n^{(n)}+
u_1'y_1^{(n-1)}+u_2'y_2^{(n-1)}+\cdots+u_n'y_n^{(n-1)}.
$$
Substituting this and  \eqref{eq:9.4.4} into \eqref{eq:9.4.1}
yields
$$
u_1Ly_1+u_2Ly_2+\cdots+u_nLy_n+P_0(x)\left(
u_1'y_1^{(n-1)}+u_2'y_2^{(n-1)}+\cdots+u_n'y_n^{(n-1)}\right)=F(x).
$$
Since  $Ly_i=0$ $(1 \leq i \leq n)$, this reduces to
$$
u_1'y_1^{(n-1)}+u_2'y_2^{(n-1)}+\cdots+u_n'y_n^{(n-1)}=\frac{F(x)}{P_0(x)}.
$$
Combining this equation with \eqref{eq:9.4.3} shows that
$$
y_p=u_1y_1+u_2y_2+\cdots+u_ny_n
$$
is a solution of \eqref{eq:9.4.1} if
$$
\begin{array}{rcl}
u'_1y_1+u'_2y_2+&\cdots&+u'_ny_n=0 \\
u'_1y'_1+u'_2y'_2+&\cdots&+u'_ny'_n=0 \\
&\vdots& \\
u'_1y_1^{(n-2)}+u'_2y^{(n-2)}_2+&\cdots&+u'_ny^{(n-2)}_n
=0 \\
u'_1y^{(n-1)}_1+u'_2y^{(n-1)}_2+&\cdots&+u'_n
y^{(n-1)}_n=F/P_0,
\end{array}
$$
which can be written in matrix form as
\begin{equation} \label{eq:9.4.5}
\begin{bmatrix}
y_1&y_2&\cdots&y_n \\
y'_1&y'_2&\cdots&y_n'\\
\vdots&\vdots&\ddots&\vdots\\
y_1^{(n-2)}&y_2^{(n-2)}&\cdots&y_n^{(n-2)}\\
y_1^{(n-1)}&y_2^{(n-1)}&\cdots&y_n^{(n-1)}
\end{bmatrix}
\begin{bmatrix}u_1'\\u_2'\\\vdots\\u_{n-1}'\\u_n'\end{bmatrix}=
\begin{bmatrix}0\\0\\ \vdots\\0\\F/ P_0\end{bmatrix}.
\end{equation}
The determinant of this system  is the Wronskian $W$ of
the fundamental set of solutions  $\{y_1,y_2,\dots,y_n\}$,
which has no zeros on $(a,b)$, by Theorem~\ref{thmtype:9.1.4}.
Solving
\eqref{eq:9.4.5} by Cramer's rule  yields
\begin{equation} \label{eq:9.4.6}
u'_j=(-1)^{n-j}\frac{FW_j}{P_0W},\quad 1\leq j\leq n,
\end{equation}
 where $W_j$ is the Wronskian of the set of
functions obtained by deleting $y_j$ from  $\{y_1,y_2,\dots,y_n\}$
 and keeping the remaining functions in the
same order. Equivalently, $W_j$ is the determinant obtained by
deleting the last row and $j$-th column of $W$.

Having obtained $u_1', u_2', \dots, u_n'$, we can integrate  to
obtain $u_1,\,u_2,\dots,u_n$. As in \href{https://ximera.osu.edu/ode/main/variationOfParameters/variationOfParameters}{Trench 5.7}, we
take the constants
of integration to be zero, and we drop any linear combination of
$\{y_1,y_2,\dots,y_n\}$ that may appear in $y_p$.


\begin{remark}
For efficiency, it's best to compute $W_1, W_2, \dots, W_n$
first, and then compute $W$ by expanding in cofactors of the last row;
thus,
$$
W=\sum_{j=1}^n(-1)^{n-j}y_j^{(n-1)}W_j.
$$
\end{remark}



\subsection*{Third Order Equations}

If $n=3$, then
$$
W=\begin{vmatrix}
y_1&y_2&y_3 \\
y'_1&y'_2&y'_3 \\
y''_1&y''_2&y''_3 \end{vmatrix}
$$
Therefore
$$
W_1=\begin{vmatrix}
y_2&y_3 \\
y'_2&y'_3 \end{vmatrix},
\quad W_2=\begin{vmatrix}
y_1&y_3 \\
 y'_1&y'_3 \end{vmatrix},
\quad W_3=\begin{vmatrix}
y_1&y_2 \\
y'_1&y'_2 \end{vmatrix},
$$
and  \eqref{eq:9.4.6}  becomes
\begin{equation} \label{eq:9.4.7}
u'_1=\frac{FW_1}{P_0W},\quad u'_2=-\frac{FW_2}{P_0W},\quad
u'_3=\frac{FW_3}{P_0W}.
\end{equation}

\begin{example}\label{example:9.4.1}
 Find a particular solution of
\begin{equation} \label{eq:9.4.8}
xy'''-y''-xy'+y=8x^2e^x,
\end{equation}
given that  $y_1=x$, $y_2=e^x$, and $y_3=e^{-x}$ form a fundamental
set of solutions of the complementary equation. Then find the general
solution of \eqref{eq:9.4.8}.
 
\begin{explanation}
We seek a particular solution of \eqref{eq:9.4.8} of the form
$$
y_p=u_1x+u_2e^x+u_3e^{-x}.
$$
The Wronskian of  $\{y_1,y_2,y_3\}$ is
$$
W(x)=\begin{vmatrix}
 x&e^x&e^{-x} \\ 1&e^x&-e^{-x} \\ 0&e^x&e^{-x}
\end{vmatrix},
$$
so
\begin{eqnarray*}
W_1&=&
\begin{vmatrix}
e^x&e^{-x}\\ e^x&-e^{-x}
\end{vmatrix}=-2,\\
W_2&=&
\begin{vmatrix}
x&e^{-x}\\1&-e^{-x}
\end{vmatrix}=-e^{-x}(x+1),\\
W_3&=&
\begin{vmatrix}
x&e^x\\1&e^x
\end{vmatrix}=e^x(x-1).
\end{eqnarray*}
Expanding $W$ by cofactors of the last row yields
$$
W=0W_1-e^x W_2+e^{-x}W_3=0(-2)-e^x\left(-e^{-x}(x+1)\right)
+e^{-x}e^x(x-1)=2x.
$$
 Since $F(x)=8x^2e^x$ and $P_0(x)=x$,
$$
\frac{F}{P_0W}=\frac{8x^2e^x}{x\cdot 2x}=4e^x.
$$
Therefore, from \eqref{eq:9.4.7}
\begin{eqnarray*}
u'_1&=& 4e^xW_1=4e^x(-2)=-8e^x,\\
u_2'&=&-4e^xW_2=-4e^x\left(-e^{-x}(x+1)\right)=4(x+1),\\
u_3'&=&4e^xW_3=4e^x\left(e^x(x-1)\right)=4e^{2x}(x-1).
\end{eqnarray*}
 Integrating and taking the constants of
integration to be zero yields
$$
u_1=-8e^x,\quad u_2=2(x+1)^2, u_3=e^{2x}(2x-3).
$$
Hence,
\begin{eqnarray*}
y_p&=&u_1y_1+u_2y_2+u_3y_3\\
&=&(-8e^x)x+e^x(2(x+1)^2)+e^{-x}\left(e^{2x}(2x-3)\right)
\\&=&e^x(2x^2-2x-1).
\end{eqnarray*}
Since $-e^x$ is a solution of the complementary equation, we redefine
$$
y_p=2xe^x(x-1).
$$
 Therefore the general solution of \eqref{eq:9.4.8} is
$$
y=2xe^x(x-1)+c_1x+c_2e^x+c_3e^{-x}.
$$
\end{explanation}
\end{example}


\subsection*{Fourth Order Equations}

If $n=4$, then
$$
W=\begin{vmatrix}
y_1&y_2&y_3&y_4 \\
y'_1&y'_2&y'_3&y_4' \\
y''_1&y''_2&y''_3&y_4''\\
y'''_1&y'''_2&y'''_3&y_4'''
 \end{vmatrix},
$$
Therefore
$$
W_1=\begin{vmatrix}
y_2&y_3&y_4 \\
y'_2&y'_3&y_4'\\
y''_2&y''_3&y_4''
\end{vmatrix},
\quad W_2=\begin{vmatrix}
y_1&y_3&y_4 \\
y'_1&y'_3&y_4'\\
y''_1&y''_3&y_4''
 \end{vmatrix},
$$

$$
 W_3=\begin{vmatrix}
y_1&y_2&y_4 \\
y'_1&y'_2&y_4'\\
y''_1&y''_2&y_4''
 \end{vmatrix},\quad
 W_4=\begin{vmatrix}
y_1&y_2&y_3 \\
y_1'&y'_2&y_3'\\
y_1''&y''_2&y_3''
 \end{vmatrix},
$$
and  \eqref{eq:9.4.6}  becomes
\begin{equation} \label{eq:9.4.9}
u'_1=-\frac{FW_1}{P_0W},\quad u'_2=\frac{FW_2}{P_0W},\quad
u'_3=-\frac{FW_3}{P_0W},\quad u'_4=\frac{FW_4}{P_0W}.
\end{equation}

\begin{example}\label{example:9.4.2}
 Find a particular solution of
\begin{equation} \label{eq:9.4.10}
x^4y^{(4)}+6x^3y'''+2x^2y''-4xy'+4y=12x^2,
\end{equation}
given that  $y_1=x$, $y_2=x^2$, $y_3=1/x$ and $y_4=1/x^2$ form a
fundamental set of solutions of the complementary equation. Then find
the general solution of \eqref{eq:9.4.10} on $(-\infty,0)$ and
$(0,\infty)$.
 

\begin{explanation}
We seek a particular solution of \eqref{eq:9.4.10} of the form
$$
y_p=u_1x+u_2x^2+\frac{u_3}{x}+\frac{u_4}{x^2}.
$$
The Wronskian of  $\{y_1,y_2,y_3,y_4\}$ is
$$
W(x)=\begin{vmatrix}
 x&x^2&1/x&-1/x^2 \\
 1&2x&-1/x^2&-2/x^3 \\
 0&2&2/x^3&6/x^4\\
0&0&-6/x^4&-24/x^5
\end{vmatrix},
$$
so
\begin{eqnarray*}
W_1&=&
\begin{vmatrix}
x^2&1/x&1/x^2\\
2x&-1/x^2&-2/x^3\\
2&2/x^3&6/x^4
\end{vmatrix}=-\frac{12}{x^4},\\
W_2&=&
\begin{vmatrix}
x&1/x&1/x^2\\
1&-1/x^2&-2/x^3\\
0&2/x^3&6/x^4
\end{vmatrix}=-\frac{6}{x^5},\\
W_3&=&
\begin{vmatrix}
x&x^2&1/x^2\\
1&2x&-2/x^3\\
0&2&6/x^4
\end{vmatrix} =\frac{12}{x^2},  \\
W_4&=&
\begin{vmatrix}
x&x^2&1/x\\
1&2x&-1/x^2\\
0&2&2/x^3
\end{vmatrix}=\frac{6}{x}.
\end{eqnarray*}
Expanding $W$ by cofactors of the last row yields
\begin{eqnarray*}
W&=&-0W_1+0 W_2-\left(-\frac{6}{x^4}\right)W_3+\left(-\frac{24}{x^5}\right)W_4\\
&=&\frac{6}{x^4}\frac{12}{x^2}-\frac{24}{x^5}\frac{6}{x}=-\frac{72}{x^6}.
\end{eqnarray*}
 Since $F(x)=12x^2$ and $P_0(x)=x^4$,
$$
\frac{F}{P_0W}=\frac{12x^2}{x^4}\left(-\frac{x^6}{72}\right)=-\frac{x^4}{6}.
$$
Therefore, from \eqref{eq:9.4.9},
\begin{eqnarray*}
u'_1&=&-\left(-\frac{x^4}{6}\right)W_1=\frac{x^4}{6}\left(-\frac{12}{x^4}\right)=-2,\\
 u_2'&=&-\frac{x^4}{6}W_2=-\frac{x^4}{6}\left(-\frac{6}{x^5}\right)
=\frac{1}{x},\\
u_3'&=&-\left(-\frac{x^4}{6}\right)W_3=\frac{x^4}{6}\frac{12}{x^2}=2x^2,\\
  u_4'&=&-\frac{x^4}{6}W_4=-\frac{x^4}{6}\frac{6}{x}=-x^3
\end{eqnarray*}
 Integrating these and taking the constants of
integration to be zero yields
$$
u_1=-2x,\quad u_2=\ln|x|,\quad u_3=\frac{2x^3}{3},
u_4=-\frac{x^4}{4}.
$$
Hence,
\begin{eqnarray*}
y_p&=&u_1y_1+u_2y_2+u_3y_3+u_4y_4\\
&=&(-2x)x+(\ln|x|)x^2+\frac{2x^3}{3}\frac{1}{x}+\left(-\frac{x^4}{4}\right)
\frac{1}{x^2} \\&=&x^2\ln|x|-\frac{19x^2}{12}.
\end{eqnarray*}
Since $-19x^2/12$ is a solution of the complementary equation, we redefine
$$
y_p=x^2\ln|x|.
$$
 Therefore
$$
y=x^2\ln|x|+c_1x+c_2x^2+\frac{c_3}{x}+\frac{c_4}{x^2}
$$
is the general solution of \eqref{eq:9.4.10} on $(-\infty,0)$ and $(0,\infty)$.
\end{explanation}
\end{example}




\section*{Text Source}
Trench, William F., "Elementary Differential Equations" (2013). Faculty Authored and Edited Books \& CDs. 8. (CC-BY-NC-SA)

\href{https://digitalcommons.trinity.edu/mono/8/}{https://digitalcommons.trinity.edu/mono/8/}


\end{document}