\documentclass{ximera}
 
%% You can put user macros here
%% However, you cannot make new environments

\listfiles

%\graphicspath{{./}{firstExample/}{secondExample/}}
\graphicspath{{./}
{aboutDiffEq/}
{applicationsLeadingToDiffEq/}
{applicationsToCurves/}
{autonomousSecondOrder/}
{basicConcepts/}
{bernoulli/}
{constCoeffHomSysI/}
{constCoeffHomSysII/}
{constCoeffHomSysIII/}
{constantCoeffWithImpulses/}
{constantCoefficientHomogeneousEquations/}
{convolution/}
{coolingActivity/}
{directionFields/}
{drainingTank/}
{epidemicActivity/}
{eulersMethod/}
{exactEquations/}
{existUniqueNonlinear/}
{frobeniusI/}
{frobeniusII/}
{frobeniusIII/}
{global.css/}
{growthDecay/}
{heatingCoolingActivity/}
{higherOrderConstCoeff/}
{homogeneousLinearEquations/}
{homogeneousLinearSys/}
{improvedEuler/}
{integratingFactors/}
{interactExperiment/}
{introToLaplace/}
{introToSystems/}
{inverseLaplace/}
{ivpLaplace/}
{laplaceTable/}
{lawOfCooling/}
{linSysOfDiffEqs/}
{linearFirstOrderDiffEq/}
{linearHigherOrder/}
{mixingProblems/}
{motionUnderCentralForce/}
{nonHomogeneousLinear/}
{nonlinearToSeparable/}
{odesInSage/}
{piecewiseContForcingFn/}
{population/}
{reductionOfOrder/}
{regularSingularPts/}
{reviewOfPowerSeries/}
{rlcCircuit/}
{rungeKutta/}
{secondLawOfMotion/}
{separableEquations/}
{seriesSolNearOrdinaryPtI/}
{seriesSolNearOrdinaryPtII/}
{simplePendulum/}
{springActivity/}
{springProblemsI/}
{springProblemsII/}
{undCoeffHigherOrderEqs/}
{undeterminedCoeff/}
{undeterminedCoeff2/}
{unitStepFunction/}
{varParHigherOrder/}
{varParamNonHomLinSys/}
{variationOfParameters/}
}


\usepackage{tikz}
\usepackage{tkz-euclide}
\usepackage{tikz-3dplot}
\usepackage{tikz-cd}
\usetikzlibrary{shapes.geometric}
\usetikzlibrary{arrows}
\usetikzlibrary{decorations.pathmorphing,patterns}
\usetikzlibrary{backgrounds} % added by Felipe
% \usetkzobj{all}   % NOT ALLOWED IN RECENT TeX's ...
\pgfplotsset{compat=1.13} % prevents compile error.

\renewcommand{\vec}[1]{\mathbf{#1}}
\newcommand{\RR}{\mathbb{R}}
\newcommand{\dfn}{\textit}
\newcommand{\dotp}{\cdot}
\newcommand{\id}{\text{id}}
\newcommand\norm[1]{\left\lVert#1\right\rVert}
\newcommand{\dst}{\displaystyle}
 
\newtheorem{general}{Generalization}
\newtheorem{initprob}{Exploration Problem}

\tikzstyle geometryDiagrams=[ultra thick,color=blue!50!black]

\usepackage{mathtools}
 
\title{Hot Potato! Cooling Activity}
 
\begin{document}
 
\begin{abstract}
An experiment involving Newton's Law of Cooling.
\end{abstract}
 
\maketitle
 
The purpose of this exercise is to deepen your understanding of Newton's Law of Heating (and Cooling) which is reviewed in Trench's text in Section 4.2.  A hands-on activity will help to supplement and apply the background theory.  Students will record and predict the temperature of a potato baking in the oven and then subsequently left out to cool.  This experimental activity requires the use of an oven thermometer, preferably a digital one that has a wire to allow readings with a closed oven door.  If you have or can borrow a thermometer, you can conduct this experiment in your own kitchen.  Particularly hungry students will opt to bake a few potatoes and also be ready with some sour cream or blue cheese dressing.  Students without access to an oven may try the same experiment with the potato submerged in boiling water.
Overview
In Section 4.2 of Trench's text, Newton's Law of Cooling is summarized a first order differential equation:
\[
T'=k(T-T_m)
\]
 
This equation stipulates that $T'$, the time rate of change of the temperature of some object, is linearly proportional to the difference between the temperature  $T$ of that object and the temperature of the medium or surroundings of the object $T_m$.  Trench notes the ideal case where  $T_m$ is perfectly constant, such as when an object is in a room or a hot oven that remains at a constant temperature.  A negative sign is placed before the "temperature decay constant" of the medium $k$, which is itself some positive number that will stipulate the relative rate of the object's temperature change.  Observe that this differential equation has a very similar appearance to one describing radioactive decay (or any natural decay problem).  In fact, heating or cooling of an object is essentially an exponential decay problem.  The quantity that decays in this case is the difference between the object and its surroundings.  This difference decays at a rate proportional to its magnitude.
 
\section*{Predicting the heating process}
 
Consider what Newton's Law of Heating would predict for a potato heated in an oven.  Examine the differential equation, and remember that $k$ and $T_m$ are constants.  At what point during the heating process will the rate of change of temperature be the smallest or the largest?  How will this rate change over time?  Which of the following plots of temperature vs. time most closely represents the behavior predicted by this law?
 
% Figure 1 here
 
\emph{Feedback for incorrect answer}: This answer is not correct.  Note that the final rate of change of temperature is too steep and the initial rate of change is too small.  The temperature should asymptotically approach the temperature of the medium because Newton's Law of Heating shows that $T'$ will get very small as $T$ approaches $T_m$.  This plot suggests that the temperature of the object might actually exceed the temperature of the surroundings, which is not possible.
 
\emph{Feedback for incorrect answer}: This answer is not correct, although it is close to the correct answer.  As the temperature of the object approaches that of the medium, the slope does in fact diminish towards zero.  Thus, this final attribute of the plot is correct.  Note that Newton's Law of Heating predicts that the initial slope will be the greatest because the initial difference between $T$ and $T_m$ is larger than after some heating has occurred.  This law does not predict any lag in the beginning of the heating process, although in reality some lag may occur if the temperature of the inside of a larger object takes a bit longer to begin heating up. 
  
\emph{Feedback for correct answer}: Correct!  The initial slope will be the greatest because the initial difference between $T$ and $T_m$ is larger than after some heating has occurred.  As the temperature of the object approaches that of the medium, the slope diminishes towards zero.  Newton's Law for Heating thus predicts that the plot of temperature vs. time should be concave downward.  Take the derivative of both sides of the differential equation to see that $T''$ is equal to -k.  Since this second derivative of temperature is a constant negative value, this corresponds the curve being concave downward.
 
  
\emph{Feedback for incorrect answer}: This answer is not correct because it does not show how the rate of temperature change will itself vary during the heating process.  Note that the final rate of change of temperature should be small and the initial rate of change should be larger.  The temperature should asymptotically approach the temperature of the medium because Newton's Law of Heating shows that $T'$ will become very small as $T$ approaches $T_m$.
 
\section*{Solving the differential equation}
 
Recall that the differential equation bears a strong resemblance to an exponential decay problem.  As with that model, separation and integration provides a method to quickly solve for temperature as a function of time:
\[
\frac{dT}{dt}=-k(T-T_m)
\]
Take a moment and attempt to solve this differential equation to find a solution for temperature as a function of time.  Don't look ahead until you have found a solution or are stuck.  You may look back to Trench's use of separation of variables in Examples 2.1.3 and 2.1.4 of his text.
    
The first step is to separate the equation, putting the dependent temperature variable on the left hand side and the independent time ($dt$) variable on the right hand side.
\[
\int\!\frac{dT}{T-T_m}=\int-k\,dt
\]
Integrating once produces a logarithm on the left side and the time variable $t$ appearing explicitly on the right side, along with a constant of integration.
\[
\ln(|T-T_m |)=-kt+C
\]
Next, we raise the natural base $e$ to the power of both sides, annihilating the logarithm.  The absolute value sign can be discarded by reversing the order of the difference because, for heating, the surrounding temperature of the oven is always greater than the temperature of the object ($T_m-T$).  This process produces a new constant $A=e^c$ which must be positive.
\[
T_m-T=Ae^{-kt}
\]
Applying an initial condition $T(0)=T_o$ allows the constant $A$ to be written in terms of the temperature of the medium and the initial temperature of the object:
\[
A=T_m-T_o
\]
The final solution gives the temperature explicitly as a function of time and the other parameters.
\[
T=T_m-(T_m-T_o ) e^{-kt}
\]
With a given oven temperature $T_m$ and an initial temperature of the object $T_o$ only the temperature decay constant of the medium $k$ needs to be specified to predict the trajectory of temperature with time.  Notice that at very long times the second term decays to zero meaning that the object's temperature will approach the temperature of the surroundings.

This solution is used to produce a plot of temperature vs. time for a potato heating from room temperature towards an oven temperature of $425^{\text{o}}$F .  Different sample values of the temperature decay constant $k$ are used.  Higher values of $k$ result in faster heating towards the final temperature.

\emph{Feedback for incorrect answer}: This answer is not correct, although it is close to the correct answer.  As the temperature of the object approaches that of the medium, the slope does in fact decreases towards zero.  Thus, this final attribute is correct.  Note that Newton's Law of Heating predicts that the initial slope will be the greatest because the initial difference between $T$ and $T_m$ is larger than after some heating has occurred.  This law does not predict any lag in the beginning of the heating process, although in reality some lag may occur if the temperature of the inside of a larger object takes a bit longer to begin heating up.  
 
\emph{Feedback for correct answer}: Correct!  The initial slope will be the greatest because the initial difference between $T$ and $T_m$ is larger than after some heating has occurred.  As the temperature of the object approaches that of the medium, the slope diminishes towards zero.  Newton's Law for Heating thus predicts that the plot of temperature vs. time should be concave downward.  Take the derivative of both sides of the differential equation to see that $T''$ is equal to -k.  Since the second derivative is a constant negative value, this corresponds the curve being concave downward.

 
\emph{Feedback for incorrect answer}: This answer is not correct because it does not show how the rate of temperature change will itself vary during the heating process.  Note that the final rate of change of temperature should be small and the initial rate of change should be larger.  The temperature should asymptotically approach the temperature of the medium because Newton's Law of Heating shows that $T'$ will become very small as $T$ approaches $T_m$.

\section*{Solving the differential equation}

Recall that the differential equation bears a strong resemblance to an exponential decay problem.  As with that model, separation and integration provides a method to quickly solve for temperature as a function of time:
\[
\frac{dT}{dt}=-k(T-T_m)
\]
Take a moment and attempt to solve this differential equation to find a solution for temperature as a function of time.  Don't look ahead until you have found a solution or are stuck.  Look back to Trench's use of separation of variables in Examples 2.1.3 and 2.1.4 of his text.
   
The first step is to separate the equation, putting the dependent temperature variable on the left hand side and the independent time ($dt$) variable on the right hand side.
\[
\int dT/((T-T_m))=\int-k\,dt
\]
Integrating once produces a logarithm on the left side and an explicit time variable on the right side, along with a constant of integration.
\[
\ln(|T-T_m |)=-kt+C
\]
Next, we raise the natural base $e$ to the power of both sides, annihilating the logarithm.  The absolute value sign can be discarded by reversing the order of the difference because, for heating, the surrounding temperature of the oven is always greater than the temperature of the object ($T_m-T$).  This process produces a new constant $A=e^c$ which must be positive.
\[
T_m-T=Ae^(-kt)
\]
Applying an initial condition $T(0)=T_o$ allows the positive constant $A$ to be written in terms of the temperature of the medium and the initial temperature of the object:
\[
A=T_m-T_o
\]
The final solution gives the temperature explicitly as a function of time and the other parameters.
\[
T=T_m-(T_m-T_o ) e^(-kt)
\]
With a given oven temperature $T_m$ and an initial temperature of the object $T_o$ only the ``temperature decay constant'' of the medium $k$ needs to be specified to predict the trajectory of temperature with time.  Notice that at very long times the second term decays to zero meaning that the object's temperature will approach the temperature of the surroundings.
This solution is used to produce a plot of temperature vs.time for a potato heating from room temperature towards an oven temperature of $425^{\text{o}}$F .  Different sample values of the temperature decay constant $k$ are used.  Higher values of $k$ result in faster heating towards the final temperature.
 
\section*{The Hot Potato Experiment}
In this experiment, we will record temperature data every minute during the first twenty minutes of heating and then use that information to estimate the temperature decay constant $k$.  This will allow us to make a prediction of the time it will take the potato to cool undisturbed until it is at a good eating temperature.
Assemble the following items in your kitchen while you preheat your oven to bake at $425^{\text{o}}$F:

\begin{itemize}
\item A wired digital cooking thermometer to allow for real-time readings without opening your oven door. In this example, an Oneida Model 31161 is used.
\item At least one medium or small baking potato such as a Russet variety. The specific type of potato is likely unimportant but you should not cut them or remove a significant quantity of the skin.  It is advisable to poke the potato with a fork or knife to allow water vapor to escape and avoid breakage in the event that steam pressure builds up inside.  The potato should initially be at room temperature.
\item Oven mitts, hot pads, or tongs for moving the potato in the hot oven.
\end{itemize}

Here, a variety of potato sizes were baked at the same time, but only the middle sized one marked with an arrow is monitored throughout the baking process.  The largest and smallest potatoes can be measured once later in the process, but this is optional.  (The largest and smallest potato were both found to be very close to the temperature of the middle potato at the end of baking.)
 
The middle potato is roughly four inches long and two inches wide.  With the temperature probe inserted into the center of a much larger potato, it might take a few minutes for the temperature to increase there.  Smaller potatoes will respond more quickly.  This process where the potato begins to heat from the outside inward could be called ``lag''.  To avoid excessive lag during the initial heating process, it is advisable to avoid using potatoes on the upper end of this size range.    

{\large \textbf{Caution}: Avoid touching the hot potato, oven, or the thermometer wire.  Use oven mitts, hot pads, or tongs to avoid a burn.}

Take the following steps to carry out this experiment:
\begin{enumerate}
\item Be sure that your temperature probe is functional with good batteries.  If left out in the open, it should read steadily at somewhere near room temperature (around $70^{\text{o}}$F).
\item Insert the temperature probe into the center of the potato.  If the potato is at room temperature, the probe reading should stay roughly constant.  
\item Before inserting the potato into the fully preheated oven, wait a couple of minutes to make sure that the potato temperature is constant.  While you do that, prepare a notebook or computer spreadsheet to record the temperature every minute for the first twenty minutes.  Setting an alarm to beep every one minute is a good way to remind you to write down the temperature, but don't lose track of the overall time.  It is not a problem if you miss a reading or two.
\item Once the oven is fully preheated and the potato temperature is stabilized, you are ready to begin.  With oven mitts, hot pads, or tongs, insert the potato into the oven and close the door over the probe wire.  It is best to keep the oven door closed during the entire course of the experiment so that the oven temperature will remain as steady as possible.
\item Record the initial temperature as time ``zero'' and for every minute thereafter to at least twenty minutes.  Tabulate and plot this data to use as described below.  You can probably do some of the analysis as the potato finishes its hour in the oven and then is left out to cool, though don't lose track of the time.
\item Continue heating the potato for a total of one hour, recording at least the last few minutes of temperature in the oven. For illustration purposes, most of the data over the hour-long heating process is included in this example.
\item Remove the potato from the oven and set it on a plate or the stovetop with the probe still inserted, recording the temperature every minute or two until the potato cools to at least $180^{\text{o}}$F.  This is the temperature at which the average person would begin to be comfortable taking a bite.  Use the data you collected in the way described below to check your prediction of the time necessary to cool.  Below, the temperature during cooling of an entire hour is included for illustration purposes.
\end{enumerate}

The following data from the first twenty minutes of this example was collected and plotted:
  
Your data and curve should be smooth but may only match this data approximately.  One noticeable feature of the plot is that the temperature increase lags at the beginning because the center of the potato, where the probe is inserted, does not immediately increase as fast as the outer portion of the potato.  This effect means that it will take longer for the potato to heat up than Newton's Law of Heating would predict.  In order to estimate the ``temperature decay constant'' $k$, we should avoid using the data from the first several minutes of the experiment where the lag is present.  Using two data points after ten minutes offers an appropriate way to make an estimate of $k$.  To do this, consider the original differential equation:
\[
\frac{dT}{dt}=-k(T-T_m)
\]
For smaller time steps, note that the change in temperature with time (the slope in the plot above) does not change very much after the initial lag period:
 

After about ten minutes,$\frac{\Delta T}{\Delta t}$ is averaging around $4.5^{\text{o}}\text{F}/\text{min}$.  Since this rate of change in temperature is roughly constant, we may replace the derivative in the differential equation with an average rate of change over some time step:
\[
\frac{dT}{dt}\approx \frac{\Delta T}{\Delta t}
\]
Rearranging the differential equation gives the following:
\[
k\approx \frac{\Delta T}{\Delta t(T_m-T)}
\]
We could make an arbitrary selection of two data points at ten and fifteen minutes, with associated temperatures of $97^{\text{o}}$F and $120^{\text{o}}$F.  These times are far enough apart to have a sizable temperature change where rounding to the nearest unit of temperature will not impact the estimate.  However, they are close enough that the slope does not change much over that time range.  Note that any selection of two time-temperature data points after the initial ten minute lag period would give a reasonable estimate.  The change in temperature and time may then be plugged into the above equation, and an average difference of the medium minus object temperature
($T_m-T$) may be used since T changes:
\[
T_m=425^{\text{o}}\text{F}
\]
\[
\text{At} T=97^{\text{o}}\text{F},\quad T_m-T=328^{\text{o}}\text{F}
\]
\[
At T=120^{\text{o}}\text{F},\quad T_m-T=305^{\text{o}}\text{F}
\]
\[
\text{Average Difference} T_m-T= (328+305)/2=316.5^{\text{o}}\text{F} 
\]
\[
k\approx ((120-97))/(15-10)(316.5) \approx 0.01453 \text{min}^{-1}
\]
If this value of $k$ is used in the solution above, the prediction overestimates the trajectory of the temperature an hour long heating process:
 
The first reason for this overestimation is the effect of the lag which is explained above.  As Newton?s Law of Heating does not predict this lag, the actual experimental data remains well below the predicted temperature.  The second reason for overestimation is the effect of water vaporization.  The potato?s temperature rises to $212^{\text{o}}\text{F}$ by the last couple minutes of the hour.  By then the temperature has leveled off in an asymptotic way, approaching this boiling point of water and falling well below the predicted temperature.  Just as with the temperature lag, Newton?s Law of Heating has no way of predicting this complicating aspect involving water vaporization.  The temperature of the potato would rise past this temperature only after all of the water had evaporated, which would take a very long time and leave a very tough and unappetizing spud.

We can utilize the same  value to estimate the time it will take the potato to cool when left out on a plate or stovetop.  At around $180^{\text{o}}\text{F}$, a person could consider taking a bite without getting burned, so this will be our target temperature.  We will assume that the same basic relationships and parameters for heating will apply with equal accuracy to the cooling process.  Newton?s Law applies equally to heating and cooling, so we can use the same differential equation and solution.  The only difference here is that the medium temperature  $T_m$ is room temperature, which is $72^{\text{o}}\text{F}$  in this example.  Now, $T_o$, the initial temperature, is the temperature out of the oven.  Zero time is considered to be when the potato is removed from the oven.  In this example, that is $212^{\text{o}}\text{F}$.  Your potato will likely be at or near this temperature when it finishes its hour in the oven.  Using the numbers particular to your experiment, plug everything into the solution and solve for $t$, which is the time (in minutes) that the potato will take to reach a target temperature of $T=180^{\text{o}}\text{F}$.
\[
T=T_m-(T_m-T_o ) e^{-kt}
\]

Using the numbers in this example, the necessary cooling time is found to be about 17.9 minutes.  Note that the potato would cool much faster if you cut or mashed it.  Increasing the surface area promotes faster cooling.  Yet cutting or mashing would significantly increase the temperature decay constant and our previous value would not apply.

In the experiment, the potato cools to the target temperature of $180^{\text{o}}\text{F}$ after about 17 minutes, which means our prediction had a high degree of accuracy as it was within one minute of the observed time. The following figure shows a very good correlation between the prediction and the experimental cooling data over an entire hour:
 
How close was your prediction?  Do you see a lag with the cooling process as well?  A small lag is noticeable in the figure above, with the potato temperature staying slightly warmer than predicted in the first several minutes.  It is remarkable that an uncut potato can stay warm for so long.  In days past, those who lived in cold climates would use hot potatoes as hand warmers (and then lunch!).  One such use of hot potatoes in coat pockets is mentioned in the novel Little House in the Big Woods by Laura Ingalls Wilder.



% insert figure here



Though we don’t yet have an estimate of k for the potato and oven, we do have an general idea of the shape and features of the heating curve.  Before we proceed with the experiment, we must discuss a complexity that has not yet been dealt with that will limit the temperature of the potato while baking.  This has to do with the significant water content within the potato itself.  Online sources suggest that it takes almost an hour to bake potatoes at an oven temperature of $425^{\text{o}}$F but that it is ideal for the potato to reach an internal temperature of $210^{\text{o}}$F.  Recall that the boiling point of water is $212^{\text{o}}$F, well below the oven temperature.  This means that the temperature of the potato will not exceed $212^{\text{o}}$F until all of the water vaporizes and leaves the potato which will not actually occur.  This is fundamentally the same phenomenon that one sees when boiling a pot of water at ambient pressure; the water temperature will remain at the boiling point during the boiling process.  Uncooked potatoes are composed mainly of water and even baked potatoes still have a significant water content.  Therefore, these practical limitations limit our potato’s temperature to below $212^{\text{o}}$F.  We do not expect the temperature of the potato to get anywhere near the temperature of the oven.
  
\section*{The Hot Potato Experiment}
In this experiment, we will record temperature every minute during the first twenty minutes of heating and then use that information to estimate the temperature decay constant $k$.  This will allow us to make a prediction of the time it will take the potato to cool - when left undisturbed - until it is at a good eating temperature.
Assemble the following items in your kitchen while you preheat your oven to bake at $425^{\text{o}}$F:
 
\begin{itemize}
\item A wired digital cooking thermometer to allow for real-time readings without opening your oven door. In this example, an Oneida Model 31161 is used.  The use of a thermometer without a wire may require opening the oven door, which will cause some fluctuation in the environment temperature.  In our simple model, this temperature ($T_m-T$) is constant.
\item At least one medium or small baking potato such as a Russet variety. The specific type of potato is likely unimportant but you should not cut them or remove a significant quantity of the skin.  It is advisable to poke the potato with a fork or knife to allow water vapor to escape and avoid breakage in the event that steam pressure builds up inside.  The potato should initially be at room temperature.
\item Oven mitts, hot pads, or tongs for moving the potato in the hot oven.
\end{itemize}
 
Here, a variety of potato sizes were baked at the same time, but only the middle sized one marked with an arrow is monitored throughout the baking process.  The largest and smallest potatoes can be measured once later in the process, but this is optional.  (The largest and smallest potato were both found to be very close to the temperature of the middle potato at the end of baking.)
  
The middle potato is roughly four inches long and two inches wide.  With the temperature probe inserted into the center of a much larger potato, it might take a few minutes for the temperature to increase there.  Smaller potatoes will respond more quickly.  This process where the potato begins to heat from the outside inward could be called ``lag''.  To avoid excessive lag during the initial heating process, it is advisable to avoid using potatoes on the upper end of this size range.   
 %need a space here
 
\begin{warning}: Avoid touching the hot potato, oven, or the thermometer wire without oven mitts, hot pads, or tongs to avoid a burn.
\end{warning}
 
Take the following steps to carry out this experiment:
\begin{enumerate}
\item Be sure that your temperature probe is functional with good batteries.  If left out in the open, it should read steadily at somewhere near room temperature (around $70^{\text{o}}$F).
\item Insert the temperature probe into the center of the potato.  If the potato is at room temperature, the probe reading should stay roughly constant. 
\item Before inserting the potato into the fully preheated oven, wait a couple of minutes to make sure that the potato temperature is constant.  While you do that, prepare a notebook or computer spreadsheet to record the temperature every minute for the first twenty minutes.  Setting an alarm to beep every one minute is a good way to remind you to write down the temperature, but don't lose track of the overall time.  It is not a problem if you miss a reading or two.
\item Once the oven is fully preheated and the potato temperature is stabilized, you are ready to begin.  With oven mitts, hot pads, or tongs, insert the potato into the oven and close the door over the probe wire.  It is best to keep the oven door closed during the entire course of the experiment so that the oven temperature will remain as steady as possible.
\item Record the initial temperature as time ``zero'' and for every minute thereafter to at least twenty minutes.  Tabulate and plot this data to use as described below.  You can probably do some of the analysis as the potato finishes its hour in the oven and then is left out to cool, though don't lose track of the time.
\item Continue heating the potato for a total of one hour, recording at least the last few minutes of temperature in the oven. For illustration purposes, most of the data over the hour-long heating process is included in this example.
\item Remove the potato from the oven and set it on a plate or the stovetop with the probe still inserted, recording the temperature every minute or two until the potato cools to at least $180^{\text{o}}$F.  This is the temperature at which the average person would begin to be comfortable taking a bite.  Use the data you collected in the way described below to check your prediction of the time necessary to cool.  Below, the temperature during cooling of an entire hour is included for illustration purposes.
\end{enumerate}
 
The following data from the first twenty minutes of this example was collected and plotted:
   
Your data and curve should be smooth but may only match this data approximately.  One noticeable feature of the plot is that the temperature increase lags at the beginning because the center of the potato, where the probe is inserted, does not immediately increase as fast as the outer portion of the potato.  This effect means that it will take longer for the potato to heat up than Newton's Law of Heating would predict.  In order to estimate the ``temperature decay constant'' $k$, we should avoid using the data from the first several minutes of the experiment where the lag is present.  Using two data points after ten minutes offers an appropriate way to make an estimate of $k$.  To do this, consider the original differential equation:
\[
\frac{dT}{dt}=-k(T-T_m)
\]
For smaller time steps, note that the change in temperature with time (the slope in the plot above) does not change very much after the initial lag period:
  
 
After about ten minutes,$\frac{\Delta T}{\Delta t}$ is averaging around $4.5^{\text{o}}\text{F}/\text{min}$.  Since this rate of change in temperature is roughly constant, we may replace the derivative in the differential equation with an average rate of change over some time step:
\[
\frac{dT}{dt}\approx \frac{\Delta T}{\Delta t}
\]
Rearranging the differential equation gives the following:
\[
k\approx \frac{\Delta T}{\Delta t(T_m-T)}
\]
We could make an arbitrary selection of two data points at ten and fifteen minutes, with associated temperatures of $97^{\text{o}}$F and $120^{\text{o}}$F.  These times are far enough apart to have a sizable temperature change where rounding to the nearest unit of temperature will not impact the estimate.  However, they are close enough that the slope does not change much over that time range.  Note that any selection of two time-temperature data points after the initial ten minute lag period would give a reasonable estimate.  The change in temperature and time may then be plugged into the above equation, and an average difference of the medium minus object temperature
($T_m-T$) may be used since T changes:
\[
T_m=425^{\text{o}}\text{F}
\]
\[
\text{At } T=97^{\text{o}}\text{F},\quad T_m-T=328^{\text{o}}\text{F}
\]
\[
\text{At } T=120^{\text{o}}\text{F},\quad T_m-T=305^{\text{o}}\text{F}
\]
\[
\text{Average Difference } T_m-T= (328+305)/2=316.5^{\text{o}}\text{F}
\]
\[
k\approx \frac{120-97}{(15-10)(316.5)} \approx 0.01453 \text{ min}^{-1}
\]
%need to reformat this above equation
If this value of $k$ is used in the solution above, the prediction overestimates the trajectory of the temperature an hour long heating process:
  
The first reason for this overestimation is the effect of the lag which is explained above.  As Newton's Law of Heating does not predict this lag, the actual experimental data remains well below the predicted temperature.  The second reason for overestimation is the effect of water vaporization.  The potato's temperature rises to $212^{\text{o}}\text{F}$ by the last couple minutes of the hour.  By then the temperature has leveled off in an asymptotic way, approaching this boiling point of water and falling well below the predicted temperature.  Just as with the temperature lag, Newton's Law of Heating has no way of predicting this complicating aspect involving water vaporization.  The temperature of the potato would rise past this temperature only after all of the water had evaporated, which would take a very long time and leave a very tough and unappetizing spud.
 
We can utilize the same  value to estimate the time it will take the potato to cool when left out on a plate or stovetop.  At around $180^{\text{o}}\text{F}$, a person could consider taking a bite without getting burned, so this will be our target temperature.  We will assume that the same basic relationships and parameters for heating will apply with equal accuracy to the cooling process.  Newton's Law applies equally to heating and cooling, so we can use the same differential equation and solution.  The only difference here is that the medium temperature  $T_m$ is room temperature, which is $72^{\text{o}}\text{F}$  in this example.  Now, $T_o$, the initial temperature, is the temperature out of the oven.  Zero time is considered to be when the potato is removed from the oven.  In this example, that is $212^{\text{o}}\text{F}$.  Your potato will likely be at or near this temperature when it finishes its hour in the oven.  Using the numbers particular to your experiment, plug everything into the solution and solve for $t$, which is the time (in minutes) that the potato will take to reach a target temperature of $T=180^{\text{o}}\text{F}$.
\[
T=T_m-(T_m-T_o ) e^{-kt}
\]
 
Using the numbers in this example, the necessary cooling time is found to be about 17.9 minutes.  Note that the potato would cool much faster if you cut or mashed it.  Increasing the surface area promotes faster cooling.  Yet cutting or mashing would significantly increase the temperature decay constant and our previous value would not apply.
 
In the experiment, the potato cools to the target temperature of $180^{\text{o}}\text{F}$ after about 17 minutes, which means our prediction had a high degree of accuracy as it was within one minute of the observed time. The following figure shows a very good correlation between the prediction and the experimental cooling data over an entire hour:
  
How close was your prediction?  Do you see a lag with the cooling process as well?  A small lag is noticeable in the figure above, with the potato temperature staying slightly warmer than predicted in the first several minutes.  It is remarkable that an uncut potato can stay warm for so long.  In days past, those who lived in cold climates would use hot potatoes as hand warmers (and then lunch!).  One such use of hot potatoes in coat pockets is mentioned in the novel Little House in the Big Woods by Laura Ingalls Wilder.
 
 
 
\end{document}
