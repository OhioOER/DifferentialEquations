\documentclass{ximera}

%% You can put user macros here
%% However, you cannot make new environments

\listfiles

%\graphicspath{{./}{firstExample/}{secondExample/}}
\graphicspath{{./}
{aboutDiffEq/}
{applicationsLeadingToDiffEq/}
{applicationsToCurves/}
{autonomousSecondOrder/}
{basicConcepts/}
{bernoulli/}
{constCoeffHomSysI/}
{constCoeffHomSysII/}
{constCoeffHomSysIII/}
{constantCoeffWithImpulses/}
{constantCoefficientHomogeneousEquations/}
{convolution/}
{coolingActivity/}
{directionFields/}
{drainingTank/}
{epidemicActivity/}
{eulersMethod/}
{exactEquations/}
{existUniqueNonlinear/}
{frobeniusI/}
{frobeniusII/}
{frobeniusIII/}
{global.css/}
{growthDecay/}
{heatingCoolingActivity/}
{higherOrderConstCoeff/}
{homogeneousLinearEquations/}
{homogeneousLinearSys/}
{improvedEuler/}
{integratingFactors/}
{interactExperiment/}
{introToLaplace/}
{introToSystems/}
{inverseLaplace/}
{ivpLaplace/}
{laplaceTable/}
{lawOfCooling/}
{linSysOfDiffEqs/}
{linearFirstOrderDiffEq/}
{linearHigherOrder/}
{mixingProblems/}
{motionUnderCentralForce/}
{nonHomogeneousLinear/}
{nonlinearToSeparable/}
{odesInSage/}
{piecewiseContForcingFn/}
{population/}
{reductionOfOrder/}
{regularSingularPts/}
{reviewOfPowerSeries/}
{rlcCircuit/}
{rungeKutta/}
{secondLawOfMotion/}
{separableEquations/}
{seriesSolNearOrdinaryPtI/}
{seriesSolNearOrdinaryPtII/}
{simplePendulum/}
{springActivity/}
{springProblemsI/}
{springProblemsII/}
{undCoeffHigherOrderEqs/}
{undeterminedCoeff/}
{undeterminedCoeff2/}
{unitStepFunction/}
{varParHigherOrder/}
{varParamNonHomLinSys/}
{variationOfParameters/}
}


\usepackage{tikz}
\usepackage{tkz-euclide}
\usepackage{tikz-3dplot}
\usepackage{tikz-cd}
\usetikzlibrary{shapes.geometric}
\usetikzlibrary{arrows}
\usetikzlibrary{decorations.pathmorphing,patterns}
\usetikzlibrary{backgrounds} % added by Felipe
% \usetkzobj{all}   % NOT ALLOWED IN RECENT TeX's ...
\pgfplotsset{compat=1.13} % prevents compile error.

\renewcommand{\vec}[1]{\mathbf{#1}}
\newcommand{\RR}{\mathbb{R}}
\newcommand{\dfn}{\textit}
\newcommand{\dotp}{\cdot}
\newcommand{\id}{\text{id}}
\newcommand\norm[1]{\left\lVert#1\right\rVert}
\newcommand{\dst}{\displaystyle}
 
\newtheorem{general}{Generalization}
\newtheorem{initprob}{Exploration Problem}

\tikzstyle geometryDiagrams=[ultra thick,color=blue!50!black]

\usepackage{mathtools}

\title{Hot Potato! Activity on heating and cooling}
\author{Anna Davis, Justin Greenly, L. Felipe Martins, Paul Zachlin}

\begin{document}

\begin{abstract}
An experiment involving Newton's Law of Cooling.
\end{abstract}

\maketitle

The purpose of this exercise is to deepen your understanding of Newton's Law of Heating (and Cooling) which is reviewed in Trench's text in Section 4.2.  A hands-on activity will help to supplement and apply the background theory.  Students will record and predict the temperature of a potato baking in the oven and then subsequently left out to cool.  This experimental activity requires the use of an oven thermometer, preferably a digital one that has a wire to allow readings with a closed oven door.  If you have or can borrow a thermometer, you can conduct this experiment in your own kitchen.  Particularly hungry students will opt to bake a few potatoes and also be ready with some sour cream or blue cheese dressing.  Students without access to an oven may try the same experiment with the potato submerged in boiling water.
Overview
In Section 4.2 of Trench's text, Newton's Law of Cooling is summarized a first order differential equation:
\[
T'=k(T-T_m)
\]
 
This equation stipulates that $T'$, the time rate of change of the temperature of some object, is linearly proportional to the difference between the temperature  $T$ of that object and the temperature of the medium or surroundings of the object $T_m$.  Trench notes the ideal case where  $T_m$ is perfectly constant, such as when an object is in a room or a hot oven that remains at a constant temperature.  A negative sign is placed before the "temperature decay constant" of the medium $k$, which is itself some positive number that will stipulate the relative rate of the object's temperature change.  Observe that this differential equation has a very similar appearance to one describing radioactive decay (or any natural decay problem).  In fact, heating or cooling of an object is essentially an exponential decay problem.  The quantity that decays in this case is the difference between the object and its surroundings.  This difference decays at a rate proportional to its magnitude.
 
\section*{Predicting the heating process}

\begin{problem}
Consider what Newton's Law of Heating would predict for a potato heated in an oven.  Examine the differential equation, and remember that k and Tm are constants.  At what point during the heating process will the rate of change of temperature be the smallest or the largest?  How will this rate change over time?  Which of the following plots of temperature vs. time most closely represents the behavior predicted by this law?


  Enter the ID number $(n=1, 2, 3, 4)$ of the correct graph. 
  $$n = \answer[format=integer,id=n]{3}$$
 
    \begin{center}  
  \begin{tikzpicture}  
    \begin{axis}[  
        xmin=0,  
        xmax=1,  
        ymin=0,  
        ymax=1,  
        ticks=none,
        %axis lines=center,  
        xlabel=time,  
        ylabel=Temperature,  
        %every axis y label/.style={at=(current axis.above origin),anchor=south},  
        %every axis x label/.style={at=(current axis.right of origin),anchor=west},  
      ]  
      \addplot [ultra thick, blue, smooth] {x^2+0.1};  
    \end{axis}  
    \node[] at (-0.5, 5.5)  (r3)    {$T_m$};
    \node[] at (-0.5, 0.6)  (r3)    {$T_0$};
    
    \node[red] at (1.5, 5.2)  (r3)    {Graph ID: $n=1$};
  \end{tikzpicture}  
\end{center}

Expand for discussion.

\begin{expandable}
    Note that the final rate of change of temperature is greater than the initial rate of change.  The temperature should asymptotically approach the temperature of the medium because Newton’s Law of Heating shows that $T'$ will get very small as $T$ approaches $T_m$.  This plot suggests that the temperature of the object might actually exceed the temperature of the surroundings, which is impossible.
 \end{expandable}

\begin{center}  
  \begin{tikzpicture}  
    \begin{axis}[  
        xmin=0,  
        xmax=2,  
        ymin=0,  
        ymax=2,  
        ticks=none,
        %axis lines=center,  
        xlabel=time,  
        ylabel=Temperature,  
        %every axis y label/.style={at=(current axis.above origin),anchor=south},  
        %every axis x label/.style={at=(current axis.right of origin),anchor=west},  
      ]  
      \addplot [ultra thick, blue, smooth] {2/(1+exp(-3.3*x+2.5))};  
    \end{axis}  
    \node[] at (-0.5, 5.5)  (r3)    {$T_m$};
    \node[] at (-0.5, 0.6)  (r3)    {$T_0$};
    
    \node[red] at (1.5, 5.2)  (r3)    {Graph ID: $n=2$};
  \end{tikzpicture}  
\end{center}

Expand for discussion.

\begin{expandable}
    As the temperature of the object approaches that of the medium, the slope of the tangent line approaches zero.  Thus, this final attribute is correct.  Note that Newton’s Law of Heating predicts that the initial slope will be the greatest because the initial difference between $T$ and $T_m$ is larger than after some heating has occurred.  This law does not predict any lag in the beginning of the heating process, although in reality some lag may occur if the temperature of the inside of a larger object takes a bit longer to begin heating up. 
 \end{expandable}

\begin{center}  
  \begin{tikzpicture}  
    \begin{axis}[  
        xmin=0,  
        xmax=5,  
        ymin=0,  
        ymax=1.1,  
        ticks=none,
        %axis lines=center,  
        xlabel=time,  
        ylabel=Temperature,  
        %every axis y label/.style={at=(current axis.above origin),anchor=south},  
        %every axis x label/.style={at=(current axis.right of origin),anchor=west},  
      ]  
      \addplot [ultra thick, blue, smooth] {1.1-2^(-x)};  
    \end{axis}  
    \node[] at (-0.5, 5.5)  (r3)    {$T_m$};
    \node[] at (-0.5, 0.6)  (r3)    {$T_0$};
    
    \node[red] at (1.5, 5.2)  (r3)    {Graph ID: $n=3$};
  \end{tikzpicture}  
\end{center}

Expand for discussion.

\begin{expandable}
    The initial slope will be the greatest because the initial difference between $T$ and $T_m$ is larger than after some heating has occurred.  As the temperature of the object approaches that of the medium, the slope approaches zero.  Newton’s Law for Heating thus predicts that the plot of temperature vs. time should be concave downward.  Take the derivative of both sides of the differential equation to see that $T''$ is equal to $–k$.  Since the second derivative is a constant negative value, the curve is concave downward.
 \end{expandable}

\begin{center}  
  \begin{tikzpicture}  
    \begin{axis}[  
        xmin=0,  
        xmax=1,  
        ymin=0,  
        ymax=1,  
        ticks=none,
        %axis lines=center,  
        xlabel=time,  
        ylabel=Temperature,  
        %every axis y label/.style={at=(current axis.above origin),anchor=south},  
        %every axis x label/.style={at=(current axis.right of origin),anchor=west},  
      ]  
      \addplot [ultra thick, blue, smooth] {0.9*x+0.1};  
    \end{axis}  
    \node[] at (-0.5, 5.5)  (r3)    {$T_m$};
    \node[] at (-0.5, 0.6)  (r3)    {$T_0$};
    
    \node[red] at (1.5, 5.2)  (r3)    {Graph ID: $n=4$};
  \end{tikzpicture}  
\end{center}

Expand for discussion.

 \begin{expandable}
    This graph does not capture how the rate of temperature change will – itself – vary during the heating process.  Note that the final rate of change of temperature should be small and the initial rate of change should be larger.  The temperature should asymptotically approach the temperature of the medium because Newton's Law of Heating shows that $T'$ will become very small as T approaches $T_m$.
  \end{expandable}
 
\end{problem}

\section*{My Hot Water Experiment}

\begin{center}  
\desmos{r3o4cg6dqm}{800}{600}  
\end{center}

\section*{Solving the differential equation}

Recall that the differential equation bears a strong resemblance to an exponential decay problem.  As with that model, separation and integration provides a method to quickly solve for temperature as a function of time:
\[
\frac{dT}{dt}=-k(T-T_m)
\]

Take a moment and attempt to solve this differential equation to find a solution for temperature as a function of time.  Don?t look ahead until you have found a solution or are stuck.  Look back to Trench?s use of ?separation of variables? in Examples 2.1.3 and 2.1.4 of his text.
   
The first step is to separate the equation, putting the dependent temperature variable on the left hand side and the independent time ($dt$)  variable on the right hand side.
\[
\int\frac{dT}{T-T_m}=\int -k\,dt
\]
Integrating once produces a logarithm on the left side and an explicit time variable on the right side, along with a constant of integration.
\[
\ln|T-T_m|=-kt+C
\]
Next, we raise the natural base $e$ to the power of both sides, annihilating the logarithm.  The absolute value sign can be discarded by reversing the order of the difference because --- for heating --- the surrounding temperature of the oven is always greater than the temperature of the object ($T_m-T$). This process produces a new constant $A=e^C$ which must be positive.
\[
T_m-T=Ae^{-kt}
\] 
Applying an initial condition $T(0)=T_0$ allows the positive constant $A$ to be written in terms of the temperature of the medium and the initial temperature of the object:
\[
A=T_m-T_0
\]
The final solution gives the temperature explicitly as a function of time and the other parameters.
\[
T=T_m-(T_m-T_0)e^{-kt}
\]

With a given oven temperature $T_m$ and an initial temperature of the object $T_0$ only the ``temperature decay constant'' of the medium  needs to be specified to predict the trajectory of temperature with time.  Notice that at very long times the second term decays to zero meaning that the object's temperature will approach the temperature of the surroundings.

This solution is used to produce a plot of temperature vs. time for a potato heating from room temperature towards an oven temperature of $425^{\text{o}}\text{F}$.  Different sample values of the temperature decay constant  are used.  Higher values of  result in faster heating towards the final temperature.

\begin{image}
\includegraphics{hotpotato_1.png}
\end{image}

Though we don't yet have an estimate of $k$ for the potato and oven, we do have a general idea of the shape and features of the heating curve.  Before we proceed with the experiment, we must discuss a complexity that has not yet been dealt with that will limit the temperature of the potato while baking.  This has to do with the significant water content within the potato itself.  Online sources suggest that it takes almost an hour to bake potatoes at an oven temperature of $425^{\text{o}}\text{F}$  but that it is ideal for the potato to reach an internal temperature of $210^{\text{o}}\text{F}$.  Recall that the boiling point of water is $212^{\text{o}}\text{F}$ , well below the oven temperature.  This means that the temperature of the potato will not exceed $210^{\text{o}}\text{F}$  until all of the water vaporizes and leaves the potato which will not actually occur. This is fundamentally the same phenomenon that one sees when boiling a pot of water at ambient pressure; the water temperature will remain at the boiling point during the boiling process.  Uncooked potatoes are composed mainly of water and even baked potatoes still have a significant water content.  Therefore, these practical limitations limit our potato?s temperature to about $212^{\text{o}}\text{F}$.  We do not expect the temperature of the potato to get anywhere near the temperature of the oven.


\section*{The Hot Potato Experiment}

In this experiment, we will record temperature data every minute during the first twenty minutes of heating and then use that information to estimate the temperature decay constant $k$.  This will allow us to make a prediction of the time it will take the potato to cool undisturbed until it is at a good eating temperature.
Assemble the following items in your kitchen while you preheat your oven to bake at $425^{\text{o}}\text{F}$:
\begin{itemize}
\item A wired digital cooking thermometer to allow for real-time readings without opening your oven door. In this example, an Oneida Model 31161 is used.
\item At least one medium or small baking potato such as a Russet variety. The specific type of potato is likely unimportant but you should not cut them or remove a significant quantity of the skin.  It is advisable to poke the potato with a fork or knife to allow water vapor to escape and avoid breakage in the event that steam pressure builds up inside.  The potato should initially be at room temperature
\item Oven mitts, hot pads, or tongs for moving the potato in the hot oven.
\end{itemize}

Here, a variety of potato sizes were baked at the same time, but only the middle sized one marked with an arrow is monitored throughout the baking process.  The largest and smallest potatoes can be measured once later in the process, but this is optional.  (The largest and smallest potato were both found to be very close to the temperature of the middle potato at the end of baking.)

\begin{image}
\includegraphics{hotpotato_2.png}
\end{image}

\end{document}
