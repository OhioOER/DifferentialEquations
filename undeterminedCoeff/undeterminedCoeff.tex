\documentclass{ximera}
 
%% You can put user macros here
%% However, you cannot make new environments

\listfiles

%\graphicspath{{./}{firstExample/}{secondExample/}}
\graphicspath{{./}
{aboutDiffEq/}
{applicationsLeadingToDiffEq/}
{applicationsToCurves/}
{autonomousSecondOrder/}
{basicConcepts/}
{bernoulli/}
{constCoeffHomSysI/}
{constCoeffHomSysII/}
{constCoeffHomSysIII/}
{constantCoeffWithImpulses/}
{constantCoefficientHomogeneousEquations/}
{convolution/}
{coolingActivity/}
{directionFields/}
{drainingTank/}
{epidemicActivity/}
{eulersMethod/}
{exactEquations/}
{existUniqueNonlinear/}
{frobeniusI/}
{frobeniusII/}
{frobeniusIII/}
{global.css/}
{growthDecay/}
{heatingCoolingActivity/}
{higherOrderConstCoeff/}
{homogeneousLinearEquations/}
{homogeneousLinearSys/}
{improvedEuler/}
{integratingFactors/}
{interactExperiment/}
{introToLaplace/}
{introToSystems/}
{inverseLaplace/}
{ivpLaplace/}
{laplaceTable/}
{lawOfCooling/}
{linSysOfDiffEqs/}
{linearFirstOrderDiffEq/}
{linearHigherOrder/}
{mixingProblems/}
{motionUnderCentralForce/}
{nonHomogeneousLinear/}
{nonlinearToSeparable/}
{odesInSage/}
{piecewiseContForcingFn/}
{population/}
{reductionOfOrder/}
{regularSingularPts/}
{reviewOfPowerSeries/}
{rlcCircuit/}
{rungeKutta/}
{secondLawOfMotion/}
{separableEquations/}
{seriesSolNearOrdinaryPtI/}
{seriesSolNearOrdinaryPtII/}
{simplePendulum/}
{springActivity/}
{springProblemsI/}
{springProblemsII/}
{undCoeffHigherOrderEqs/}
{undeterminedCoeff/}
{undeterminedCoeff2/}
{unitStepFunction/}
{varParHigherOrder/}
{varParamNonHomLinSys/}
{variationOfParameters/}
}


\usepackage{tikz}
\usepackage{tkz-euclide}
\usepackage{tikz-3dplot}
\usepackage{tikz-cd}
\usetikzlibrary{shapes.geometric}
\usetikzlibrary{arrows}
\usetikzlibrary{decorations.pathmorphing,patterns}
\usetikzlibrary{backgrounds} % added by Felipe
% \usetkzobj{all}   % NOT ALLOWED IN RECENT TeX's ...
\pgfplotsset{compat=1.13} % prevents compile error.

\renewcommand{\vec}[1]{\mathbf{#1}}
\newcommand{\RR}{\mathbb{R}}
\newcommand{\dfn}{\textit}
\newcommand{\dotp}{\cdot}
\newcommand{\id}{\text{id}}
\newcommand\norm[1]{\left\lVert#1\right\rVert}
\newcommand{\dst}{\displaystyle}
 
\newtheorem{general}{Generalization}
\newtheorem{initprob}{Exploration Problem}

\tikzstyle geometryDiagrams=[ultra thick,color=blue!50!black]

\usepackage{mathtools}
%\input{../../../../fimacros.tex}
 
 
 
\title{5.4 The Method of Undetermined Coefficients I}
 
 
\begin{document}
 
\begin{abstract}
 We explore the solution of nonhomogeneous linear equations in the case where the forcing function is the product of an exponential function and a polynomial.
\end{abstract}
 
\maketitle
 
\section*{The Method of Undetermined Coefficients I}
 
In this section we consider the constant coefficient equation
\begin{equation} \label{eq:5.4.1}
ay''+by'+cy=e^{\alpha x}G(x),
\end{equation}
where $\alpha$ is a constant and $G$ is a polynomial.
 
 
From Theorem~\ref{thmtype:5.3.2}, the general solution of (\ref{eq:5.4.1})
is
$y=y_p+c_1y_1+c_2y_2$, where $y_p$ is a particular solution of
(\ref{eq:5.4.1}) and $\{y_1,y_2\}$ is a fundamental set of
solutions of the complementary equation
$$
ay''+by'+cy=0.
$$
In \href{https://ximera.osu.edu/ode/main/constantCoefficientHomogeneousEquations/constantCoefficientHomogeneousEquations}{Trench 5.2} we showed how to find $\{y_1,y_2\}$. In this
section we'll show how to find $y_p$. The procedure that we'll use
is called \textit{the method of undetermined coefficients}.
 
%Our first example is similar to
%Exercises~\ref{exer:5.3.16}--\ref{exer:5.3.21}.
 
\begin{example}\label{example:5.4.1}
Find a particular solution of
\begin{equation} \label{eq:5.4.2}
y''-7y'+12y=4e^{2x}.
\end{equation}
Then find the general solution.
 
 
\begin{explanation}
Substituting $y_p=Ae^{2x}$ for $y$ in (\ref{eq:5.4.2}) will produce a
constant multiple of $Ae^{2x}$ on the left side of (\ref{eq:5.4.2}), so it
may be possible to choose $A$ so that $y_p$ is a solution of
(\ref{eq:5.4.2}). Let's try it;   if $y_p=Ae^{2x}$ then
$$
y_p''-7y_p'+12y_p=4Ae^{2x}-14Ae^{2x}+12Ae^{2x}=2Ae^{2x}=4e^{2x}
$$
if $A=2$. Therefore $y_p=2e^{2x}$ is a particular solution of
(\ref{eq:5.4.2}). To find the general solution, we note that the
characteristic polynomial of the complementary equation
\begin{equation} \label{eq:5.4.3}
y''-7y'+12y=0
\end{equation}
is $p(r)=r^2-7r+12=(r-3)(r-4)$, so $\{e^{3x},e^{4x}\}$ is a
fundamental set of solutions of (\ref{eq:5.4.3}). Therefore the general
solution of (\ref{eq:5.4.2}) is
$$
 y=2e^{2x}+c_1e^{3x}+c_2e^{4x}.
$$
 
\end{explanation}
\end{example}
 
\begin{example}\label{example:5.4.2}
Find a particular solution of
\begin{equation} \label{eq:5.4.4}
y''-7y'+12y=5e^{4x}.
\end{equation}
Then find the general solution.
 
\begin{explanation}
Fresh from our success in finding a particular solution of
(\ref{eq:5.4.2}) --- where we chose $y_p=Ae^{2x}$ because the right side
of
(\ref{eq:5.4.2}) is a constant multiple of $e^{2x}$ --- it may seem
reasonable to try $y_p=Ae^{4x}$ as a particular solution of
(\ref{eq:5.4.4}). However, this won't work, since we saw in
Example~\ref{example:5.4.1} that $e^{4x}$ is a solution of the
complementary equation (\ref{eq:5.4.3}), so substituting $y_p=Ae^{4x}$
into the left side of (\ref{eq:5.4.4}) produces zero on the left, no
matter how we choose $A$. To discover a suitable form for $y_p$,
we use the same approach that we used in \href{https://ximera.osu.edu/ode/main/constantCoefficientHomogeneousEquations/constantCoefficientHomogeneousEquations}{Trench 5.2} to find a
second solution of
$$
ay''+by'+cy=0
$$
in the case where the characteristic equation has a repeated real
root: we look for solutions of (\ref{eq:5.4.4}) in the form $y=ue^{4x}$,
where $u$ is a function to be determined. Substituting
\begin{equation} \label{eq:5.4.5}
y=ue^{4x},\quad
y'=u'e^{4x}+4ue^{4x},\quad\mbox{and}\quad
y''=u''e^{4x}+8u'e^{4x}+16ue^{4x}
\end{equation}
into (\ref{eq:5.4.4}) and canceling the common factor $e^{4x}$ yields
$$
(u''+8u'+16u)-7(u'+4u)+12u=5,
$$
or
$$
u''+u'=5.
$$
By inspection we  see that $u_p=5x$ is a particular solution of
this equation, so $y_p=5xe^{4x}$ is a particular solution of
(\ref{eq:5.4.4}). Therefore
$$
y=5xe^{4x}+c_1e^{3x}+c_2e^{4x}
$$
is the general solution.
\end{explanation}
\end{example}
 
\begin{example}\label{example:5.4.3}
Find a particular solution of
\begin{equation} \label{eq:5.4.6}
y''-8y'+16y=2e^{4x}.
\end{equation}
 
\begin{explanation}
Since the characteristic polynomial of the complementary equation
\begin{equation} \label{eq:5.4.7}
y''-8y'+16y=0
\end{equation}
is $p(r)=r^2-8r+16=(r-4)^2$, both $y_1=e^{4x}$ and $y_2=xe^{4x}$ are
solutions of (\ref{eq:5.4.7}). Therefore (\ref{eq:5.4.6}) does not have a
solution of the form $y_p=Ae^{4x}$ or $y_p=Axe^{4x}$. As in
Example~\ref{example:5.4.2}, we look for solutions of (\ref{eq:5.4.6}) in the
form $y=ue^{4x}$, where $u$ is a function to be determined.
Substituting from (\ref{eq:5.4.5}) into (\ref{eq:5.4.6}) and canceling the
common factor $e^{4x}$ yields
$$
(u''+8u'+16u)-8(u'+4u)+16u=2,
$$
or
$$
u''=2.
$$
Integrating twice and taking the constants of integration to be zero
shows that $u_p=x^2$ is a particular solution of this equation, so
$y_p=x^2e^{4x}$ is a particular solution of (\ref{eq:5.4.4}). Therefore
$$
y=e^{4x}(x^2+c_1+c_2x)
$$
is the general solution.
\end{explanation}
\end{example}
 
The preceding examples illustrate the following facts concerning
the form of a particular solution $y_p$
of a constant coefficent equation
$$
ay''+by'+cy=ke^{\alpha x},
$$
where $k$ is a nonzero constant:
 
\begin{enumerate}
\item\label{item:ypa} % (a)
If $e^{\alpha x}$ isn't  a solution of the complementary
equation
\begin{equation} \label{eq:5.4.8}
ay''+by'+cy=0,
\end{equation}
then  $y_p=Ae^{\alpha x}$, where $A$ is a constant. (See
Example~\ref{example:5.4.1}).
\item \label{item:ypb} % (b)
If $e^{\alpha x}$ is a solution of (\ref{eq:5.4.8}) but $xe^{\alpha x}$
is not, then  $y_p=Axe^{\alpha x}$,  where $A$ is a constant.
(See Example~\ref{example:5.4.2}.)
\item \label{item:ypc}% (c)
If both $e^{\alpha x}$ and $xe^{\alpha x}$ are solutions of (\ref{eq:5.4.8}),
then  $y_p=Ax^2e^{\alpha x}$,  where $A$ is a constant.
(See Example~\ref{example:5.4.3}.)
\end{enumerate}
 
%See Exercise~\ref{exer:5.4.30}  for the proofs of these facts.
 
In all three cases you can just substitute the appropriate form for
$y_p$ and its derivatives directly into
$$
ay_p''+by_p'+cy_p=ke^{\alpha x},
$$
and solve for the constant $A$, as we did in
Example~\ref{example:5.4.1}.
% (See Exercises~\ref{exer:5.4.31}--\ref{exer:5.4.33}.)
However, if the equation is
$$
ay''+by'+cy=k e^{\alpha x}G(x),
$$
where $G$ is a polynomial of degree greater than zero, we recommend
that you use the substitution $y=ue^{\alpha x}$ as we did in
Examples~\ref{example:5.4.2} and \ref{example:5.4.3}. The equation for $u$
will turn out to be
\begin{equation} \label{eq:5.4.9}
au''+p'(\alpha)u'+p(\alpha)u=G(x),
\end{equation}
where $p(r)=ar^2+br+c$ is the characteristic polynomial of the
complementary equation and $p'(r)=2ar+b$ %(Exercise~\ref{exer:5.4.30});
however, you shouldn't memorize this since it's easy to derive the
equation for $u$ in any particular case. Note, however, that if
$e^{\alpha x}$ is a solution of the complementary equation then
$p(\alpha)=0$, so (\ref{eq:5.4.9}) reduces to
$$
au''+p'(\alpha)u'=G(x),
$$
while if both $e^{\alpha x}$ and $xe^{\alpha x}$ are solutions of the
complementary equation then $p(r)=a(r-\alpha)^2$ and
$p'(r)=2a(r-\alpha)$, so $p(\alpha)=p'(\alpha)=0$ and (\ref{eq:5.4.9})
reduces to
$$
au''=G(x).
$$
 
\begin{example}\label{example:5.4.4}
Find a particular solution of
\begin{equation} \label{eq:5.4.10}
y''-3y'+2y=e^{3x}(-1+2x+x^2).
\end{equation}
 
\begin{explanation}
Substituting
$$
y=ue^{3x},\quad y'=u'e^{3x}+3ue^{3x},\quad\mbox{and}\quad
y''=u''e^{3x}+6u'e^{3x}+9ue^{3x}
$$
into (\ref{eq:5.4.10}) and  canceling $e^{3x}$  yields
$$
(u''+6u'+9u)-3(u'+3u)+2u=-1+2x+x^2,
$$
or
\begin{equation} \label{eq:5.4.11}
u''+3u'+2u=-1+2x+x^2.
\end{equation}
As in Example~\ref{exer:5.3.2}, in order to guess a form for a particular
solution of
(\ref{eq:5.4.11}), we note that substituting a second degree polynomial
$u_p=A+Bx+Cx^2$ for $u$ in the left side of (\ref{eq:5.4.11}) produces
another second degree polynomial with coefficients that depend upon
$A$, $B$, and $C$;   thus,
$$
\mbox{if}\quad u_p=A+Bx+Cx^2\quad\mbox{then}\quad
u_p'=B+2Cx\quad\mbox{and}\quad u_p''=2C.
$$
If $u_p$ is to satisfy (\ref{eq:5.4.11}),  we must have
\begin{eqnarray*}
u_p''+3u_p'+2u_p&=&2C+3(B+2Cx)+2(A+Bx+Cx^2)\\
&=&(2C+3B+2A)+(6C+2B)x+2Cx^2=-1+2x+x^2.
\end{eqnarray*}
Equating  coefficients of like powers of $x$ on the two sides of the
last equality yields
$$
\begin{array}{rcr}
2C&=&1\\
2B+6C&=&2\\
2A+3B+2C&=& -1.
\end{array}
$$
Solving these equations for $C$, $B$, and $A$ (in that order) yields
 $C=1/2,B=-1/2,A=-1/4$.  Therefore
$$
u_p=-\frac{1}{4}(1+2x-2x^2)
$$
is a particular solution of  (\ref{eq:5.4.11}), and
$$
y_p=u_pe^{3x}=-\frac{e^{3x}}{4}(1+2x-2x^2)
$$
is a particular solution of  (\ref{eq:5.4.10}).
\end{explanation}
\end{example}
 
\begin{example}\label{example:5.4.5}
 Find a particular solution of
\begin{equation} \label{eq:5.4.12}
y''-4y'+3y=e^{3x}(6+8x+12x^2).
\end{equation}
 
\begin{explanation}
Substituting
$$
y=ue^{3x},\quad y'=u'e^{3x}+3ue^{3x},\quad\mbox{and}\quad
y''=u''e^{3x}+6u'e^{3x}+9ue^{3x}
$$
into (\ref{eq:5.4.12}) and  canceling $e^{3x}$  yields
$$
(u''+6u'+9u)-4(u'+3u)+3u=6+8x+12x^2,
$$
or
\begin{equation} \label{eq:5.4.13}
u''+2u'=6+8x+12x^2.
\end{equation}
There's no $u$ term in this equation, since $e^{3x}$ is a solution of
the complementary equation for (\ref{eq:5.4.12}).
%(SeeExercise~\ref{exer:5.4.30}.)
Therefore (\ref{eq:5.4.13}) does not have a
particular solution of the form $u_p=A+Bx+Cx^2$ that we used
successfully in Example~\ref{example:5.4.4}, since with this choice of
$u_p$,
$$
u_p''+2u_p'=2C+(B+2Cx)
$$
can't contain the last term ($12x^2$) on the right side of
(\ref{eq:5.4.13}). Instead, let's try $u_p=Ax+Bx^2+Cx^3$ on the grounds
that
$$
u_p'=A+2Bx+3Cx^2\quad\mbox{and}\quad
u_p''=2B+6Cx
$$
together contain all the powers of $x$ that appear on the right side
of (\ref{eq:5.4.13}).
 
Substituting these expressions in place of $u'$ and $u''$  in
(\ref{eq:5.4.13}) yields
$$
(2B+6Cx)+2(A+2Bx+3Cx^2)=(2B+2A)+(6C+4B)x+6Cx^2=6+8x+12x^2.
$$
Comparing coefficients of like powers of $x$ on the two sides of the
last equality shows that $u_p$ satisfies (\ref{eq:5.4.13}) if
$$
\begin{array}{rcr}
6C&=&12\\
4B+6C&=&8\\
2A+2B&=&6.
\end{array}
$$
Solving these equations successively yields $C=2$, $B=-1$, and $A=4$.
Therefore
$$
u_p=x(4-x+2x^2)
$$
is a particular solution of  (\ref{eq:5.4.13}),
and
$$
y_p=u_pe^{3x}=xe^{3x}(4-x+2x^2)
$$
is a particular solution of  (\ref{eq:5.4.12}).
\end{explanation}
\end{example}
 
\begin{example}\label{example:5.4.6}
Find a particular solution of
\begin{equation} \label{eq:5.4.14}
4y''+4y'+y=e^{-x/2}(-8+48x+144x^2).
\end{equation}
 
 
\begin{explanation}  Substituting
$$
y=ue^{-x/2},\quad  y'=u'e^{-x/2}-\frac{1}{2}ue^{-x/2},\quad\mbox{and}\quad y''=u''e^{-x/2}-u'e^{-x/2}+\frac{1}{4}ue^{-x/2}
$$
into (\ref{eq:5.4.14}) and  canceling $e^{-x/2}$  yields
$$
4\left(u''-u'+\frac{u}{4}\right)+4\left(u'-\frac{u}{2}\right)+u=4u''=-8+48x+144x^2,
$$
or
\begin{equation} \label{eq:5.4.15}
u''=-2+12x+36x^2,
\end{equation}
which does not contain $u$ or $u'$ because $e^{-x/2}$ and $xe^{-x/2}$
are both solutions of the complementary equation.
%(See Exercise~\ref{exer:5.4.30}.)
To obtain a particular solution of
(\ref{eq:5.4.15}) we integrate twice, taking the constants of integration
to be zero;   thus,
$$
u_p'=-2x+6x^2+12x^3\quad\mbox{and}\quad
u_p=-x^2+2x^3+3x^4=x^2(-1+2x+3x^2).
$$
Therefore
$$
y_p=u_pe^{-x/2}=x^2e^{-x/2}(-1+2x+3x^2)
$$
is a particular solution of  (\ref{eq:5.4.14}).
\end{explanation}
\end{example}
 
\subsection*{Summary}
 
The preceding examples illustrate the following facts concerning
 particular solutions of a constant coefficient equation of the form
$$
ay''+by'+cy=e^{\alpha x}G(x),
$$
where $G$ is a polynomial
%(see Exercise~\ref{exer:5.4.30}):
\begin{enumerate}
\item\label{item:partsolutionfactsa} % (a)
If $e^{\alpha x}$ isn't  a solution of the complementary
equation
\begin{equation} \label{eq:5.4.16}
ay''+by'+cy=0,
\end{equation}
then  $y_p=e^{\alpha x}Q(x)$,
 where $Q$ is a polynomial of the same degree as $G$.
(See Example~\ref{example:5.4.4}).
\item\label{item:partsolutionfactsb} % (b)
If $e^{\alpha x}$ is a solution of (\ref{eq:5.4.16}) but $xe^{\alpha x}$
is not, then  $y_p=xe^{\alpha x}Q(x)$,
 where $Q$ is a polynomial of the same degree as $G$.
(See Example~\ref{example:5.4.5}.)
\item\label{item:partsolutionfactsc} % (c)
If both $e^{\alpha x}$ and $xe^{\alpha x}$ are solutions of (\ref{eq:5.4.16}),
then  $y_p=x^2e^{\alpha x}Q(x)$,
 where $Q$ is a polynomial of the same degree as $G$.
(See Example~\ref{example:5.4.6}.)
\end{enumerate}
 
In all three cases, you can just substitute the appropriate form for
$y_p$ and its derivatives directly into
$$
ay_p''+by_p'+cy_p=e^{\alpha x}G(x),
$$
and solve for the coefficients of the polynomial $Q$. However, if you
try this you will see that the computations are more tedious than
those that you encounter by making the substitution $y=ue^{\alpha x}$
and finding a particular solution of the resulting equation for $u$.
%(See Exercises~\ref{exer:5.4.34}-\ref{exer:5.4.36}.)
In Case \ref{item:partsolutionfactsa}
the equation for $u$ will be of the form
$$
au''+p'(\alpha)u'+p(\alpha)u=G(x),
$$
with a particular solution of the form $u_p=Q(x)$, a polynomial of the
same degree as $G$, whose coefficients can be found by the method used
in Example~\ref{example:5.4.4}. In Case \ref{item:partsolutionfactsb}  the equation for
$u$ will be of the form
$$
au''+p'(\alpha)u'=G(x)
$$
(no $u$ term on the left), with a particular solution of the form
$u_p=xQ(x)$, where $Q$ is a polynomial of the same degree as $G$ whose
coefficents can be found by the method used in
Example~\ref{example:5.4.5}. In Case \ref{item:partsolutionfactsc}  the equation for $u$
will be of the form
$$
au''=G(x)
$$
with a particular solution of the form $u_p=x^2Q(x)$ that can be
obtained by integrating $G(x)/a$ twice and taking the constants of
integration to be zero, as in Example~\ref{example:5.4.6}.
 
\subsection*{Using the Principle of Superposition}
 
The next example shows how to combine the method of undetermined
coefficients and Theorem~\ref{thmtype:5.3.3}, the principle of
superposition.
 
\begin{example}\label{example:5.4.7}
Find a particular solution of
\begin{equation} \label{eq:5.4.17}
y''-7y'+12y=4e^{2x}+5e^{4x}.
\end{equation}
 
 
\begin{explanation}
In Example~\ref{example:5.4.1} we found that $y_{p_1}=2e^{2x}$
is a particular solution of
$$
y''-7y'+12y=4e^{2x},
$$
and in Example~\ref{example:5.4.2} we found that $y_{p_2}=5xe^{4x}$
is a particular solution of
$$
y''-7y'+12y=5e^{4x}.
$$
Therefore the principle of superposition implies that
$y_p=2e^{2x}+5xe^{4x}$ is a particular solution of (\ref{eq:5.4.17}).
\end{explanation}
\end{example}
\section*{Text Source}
Trench, William F., "Elementary Differential Equations" (2013). Faculty Authored and Edited Books \& CDs. 8. (CC-BY-NC-SA)
 
\href{https://digitalcommons.trinity.edu/mono/8/}{https://digitalcommons.trinity.edu/mono/8/}
 
\end{document}