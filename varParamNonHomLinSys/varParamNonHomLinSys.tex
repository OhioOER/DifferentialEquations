\documentclass{ximera}

%% You can put user macros here
%% However, you cannot make new environments

\listfiles

%\graphicspath{{./}{firstExample/}{secondExample/}}
\graphicspath{{./}
{aboutDiffEq/}
{applicationsLeadingToDiffEq/}
{applicationsToCurves/}
{autonomousSecondOrder/}
{basicConcepts/}
{bernoulli/}
{constCoeffHomSysI/}
{constCoeffHomSysII/}
{constCoeffHomSysIII/}
{constantCoeffWithImpulses/}
{constantCoefficientHomogeneousEquations/}
{convolution/}
{coolingActivity/}
{directionFields/}
{drainingTank/}
{epidemicActivity/}
{eulersMethod/}
{exactEquations/}
{existUniqueNonlinear/}
{frobeniusI/}
{frobeniusII/}
{frobeniusIII/}
{global.css/}
{growthDecay/}
{heatingCoolingActivity/}
{higherOrderConstCoeff/}
{homogeneousLinearEquations/}
{homogeneousLinearSys/}
{improvedEuler/}
{integratingFactors/}
{interactExperiment/}
{introToLaplace/}
{introToSystems/}
{inverseLaplace/}
{ivpLaplace/}
{laplaceTable/}
{lawOfCooling/}
{linSysOfDiffEqs/}
{linearFirstOrderDiffEq/}
{linearHigherOrder/}
{mixingProblems/}
{motionUnderCentralForce/}
{nonHomogeneousLinear/}
{nonlinearToSeparable/}
{odesInSage/}
{piecewiseContForcingFn/}
{population/}
{reductionOfOrder/}
{regularSingularPts/}
{reviewOfPowerSeries/}
{rlcCircuit/}
{rungeKutta/}
{secondLawOfMotion/}
{separableEquations/}
{seriesSolNearOrdinaryPtI/}
{seriesSolNearOrdinaryPtII/}
{simplePendulum/}
{springActivity/}
{springProblemsI/}
{springProblemsII/}
{undCoeffHigherOrderEqs/}
{undeterminedCoeff/}
{undeterminedCoeff2/}
{unitStepFunction/}
{varParHigherOrder/}
{varParamNonHomLinSys/}
{variationOfParameters/}
}


\usepackage{tikz}
\usepackage{tkz-euclide}
\usepackage{tikz-3dplot}
\usepackage{tikz-cd}
\usetikzlibrary{shapes.geometric}
\usetikzlibrary{arrows}
\usetikzlibrary{decorations.pathmorphing,patterns}
\usetikzlibrary{backgrounds} % added by Felipe
% \usetkzobj{all}   % NOT ALLOWED IN RECENT TeX's ...
\pgfplotsset{compat=1.13} % prevents compile error.

\renewcommand{\vec}[1]{\mathbf{#1}}
\newcommand{\RR}{\mathbb{R}}
\newcommand{\dfn}{\textit}
\newcommand{\dotp}{\cdot}
\newcommand{\id}{\text{id}}
\newcommand\norm[1]{\left\lVert#1\right\rVert}
\newcommand{\dst}{\displaystyle}
 
\newtheorem{general}{Generalization}
\newtheorem{initprob}{Exploration Problem}

\tikzstyle geometryDiagrams=[ultra thick,color=blue!50!black]

\usepackage{mathtools}

\title{Variation of Parameters for Nonhomogeneous Linear Systems}%\label{Module 7-ADEF}


\begin{document}

\begin{abstract}

\end{abstract}

\maketitle

\section*{Variation of Parameters for Nonhomogeneous Linear Systems}

We now consider the nonhomogeneous linear system
$$
{\bf y}'= A(t){\bf y}+{\bf f}(t),
$$
where $A$ is an $n\times n$ matrix function and ${\bf f}$ is an
$n$-vector forcing function. Associated with this system is the \dfn{complementary system} ${\bf y}'=A(t){\bf y}$.

The next theorem is analogous to
Theorems~\ref{thmtype:5.3.2} and
\ref{thmtype:9.1.5}. It shows how to find the general solution
of
${\bf y}'=A(t){\bf y}+{\bf f}(t)$ if we know a particular solution of ${\bf
y}'=A(t){\bf y}+{\bf f}(t)$ and a fundamental set of solutions of the
complementary system. We leave the proof to the reader.
%as an exercise (Exercise~\ref{exer:10.7.21}).

\begin{theorem}\label{thmtype:10.7.1}
Suppose the $n\times n$ matrix function $A$ and the $n$-vector
function ${\bf f}$ are continuous on $(a,b)$. Let ${\bf y}_p$ be a
particular solution of ${\bf y}'=A(t){\bf y}+{\bf f}(t)$ on $(a,b)$,
and let
$\{{\bf y}_1,{\bf y}_2,\dots,{\bf y}_n\}$ be a fundamental set of
solutions of the complementary equation ${\bf y}'=A(t){\bf y}$ on
$(a,b)$. Then ${\bf y}$ is a solution of ${\bf y}'=A(t){\bf y}+{\bf
f}(t)$ on $(a,b)$ if and only if
$$
{\bf y}={\bf y}_p+c_1{\bf y}_1+c_2{\bf y}_2+\cdots+c_n{\bf y}_n,
$$
where $c_1, c_2, \dots, c_n$  are constants.
\end{theorem}

\subsection*{Finding a Particular Solution of a Nonhomogeneous System}

We now discuss an extension of the method of variation of parameters
to linear nonhomogeneous systems. This method will produce a
particular solution of a nonhomogenous system ${\bf y}'=A(t){\bf y}+{\bf f}(t)$ provided that we know a fundamental matrix for the
complementary system. To derive the method, suppose $Y$ is a
fundamental matrix for the complementary system; that is,
$$
Y=\begin{bmatrix}
y_{11}&y_{12}&\cdots&y_{1n} \\
y_{21}&y_{22}&\cdots&y_{2n}\\
\vdots&\vdots&\ddots&\vdots \\
y_{n1}&y_{n2}&\cdots&y_{nn} \\
\end{bmatrix},
$$
where
$$
{\bf y}_1=\begin{bmatrix}y_{11}\\y_{21}\\ \vdots\\
y_{n1}\end{bmatrix},\quad
{\bf y}_2=\begin{bmatrix}y_{12}\\y_{22}\\ \vdots\\
y_{n2}\end{bmatrix},\quad \cdots,\quad
{\bf y}_n=\begin{bmatrix}y_{1n}\\y_{2n}\\ \vdots\\
y_{nn}\end{bmatrix}
$$
is a fundamental set of solutions of the complementary system.
In Section~10.3 we saw that $Y'=A(t)Y$. We seek a
particular solution of
\begin{equation} \label{eq:10.7.1}
{\bf y}'=A(t){\bf y}+{\bf f}(t)
\end{equation}
 of the form
\begin{equation} \label{eq:10.7.2}
{\bf y}_p=Y{\bf u},
\end{equation}
 where ${\bf u}$ is to be determined. Differentiating \eqref{eq:10.7.2} yields
\begin{eqnarray*}
 {\bf y}_p'&=&Y' {\bf u}+Y {\bf u}'
\\&=&A Y {\bf u}+Y {\bf u}'\mbox{ (since $Y'=AY$)}\\&=&
 A{\bf y}_p+Y {\bf u}'\mbox{ (since $Y{\bf u}={\bf y}_p$)}.
\end{eqnarray*}
Comparing this with  \eqref{eq:10.7.1} shows that ${\bf y}_p=Y{\bf u}$ is a
solution of \eqref{eq:10.7.1} if and only if
$$
Y{\bf u}'={\bf f}.
$$
Thus, we can find a particular solution ${\bf y}_p$ by solving this
equation for ${\bf u}'$, integrating to obtain ${\bf u}$, and
computing $Y{\bf u}$. We can take all constants of integration to be
zero, since any particular solution will suffice.

%Exercise~\ref{exer:10.7.22} sketches a proof that 
This method is analogous
to the method of variation of parameters discussed in
Sections~5.7 and 9.4 for scalar linear
equations.

\begin{example}\label{example:10.7.1} 
\begin{enumerate}
\item\label{item:10.7.1a} % (a)
Find a particular solution of the system
\begin{equation} \label{eq:10.7.3}
{\bf y}'=\begin{bmatrix}1&2\\2&1\end{bmatrix}{\bf y}+\begin{bmatrix}2e^{4t}\\e^{4t}
\end{bmatrix},
\end{equation}
which we considered in Example~\ref{example:10.2.1}.
\item\label{item:10.7.1b} % (b)
Find the general solution of  \eqref{eq:10.7.3}.
\end{enumerate}

\begin{explanation}
\ref{item:10.7.1a} The complementary system is
\begin{equation} \label{eq:10.7.4}
{\bf y}'=\begin{bmatrix}1&2\\2&1\end{bmatrix}{\bf y}.
\end{equation}
The characteristic polynomial of the coefficient matrix is
$$
\begin{vmatrix}1-\lambda&2\\2&1-\lambda\end{vmatrix}=
(\lambda+1)(\lambda-3).
$$
Using the method of Section~10.4, we find that
$$
{\bf y}_1=\begin{bmatrix}e^{3t}\\e^{3t}\end{bmatrix}
\quad\mbox{and}\quad
{\bf y}_2=\begin{bmatrix}e^{-t}\\-e^{-t}\end{bmatrix}
$$
are linearly independent solutions of \eqref{eq:10.7.4}.
Therefore
$$
Y=\begin{bmatrix}e^{3t}&e^{-t}\\e^{3t}&-e^{-t}\end{bmatrix}
$$
is a fundamental matrix for \eqref{eq:10.7.4}. We seek a particular
solution ${\bf y}_p=Y{\bf u}$ of \eqref{eq:10.7.3}, where $Y{\bf u}'={\bf f}$; that is,
$$
\begin{bmatrix}e^{3t}&e^{-t}\\e^{3t}&-e^{-t}\end{bmatrix}
\begin{bmatrix}u_1'\\u_2'\end{bmatrix}
=\begin{bmatrix}2e^{4t}\\e^{4t}\end{bmatrix}.
$$
The determinant of $Y$ is the Wronskian
$$
\begin{vmatrix}e^{3t}&e^{-t}\\e^{3t}&-e^{-t}\end{vmatrix}
=-2e^{2t}.
$$
By Cramer's rule,
$$
\begin{array}{ccccccl}
u_1'&=&-\frac{1}{e^{2t}}
\begin{vmatrix}2e^{4t}&e^{-t}\\e^{4t}&-e^{-t}
\end{vmatrix}&=&\frac{3e^{3t}}{2e^{2t}}&=
&\frac{3}{2}e^t,\\
u_2'&=&-\frac{1}{2e^{2t}}
\begin{vmatrix}e^{3t}&2e^{4t}\\e^{3t}&e^{4t}
\end{vmatrix}&=&\frac{e^{7t}}{2e^{2t}}&=&\frac{1}{2}e^{5t}.
\end{array}
$$
 Therefore
$$
{\bf
u}'=\frac{1}{2}\begin{bmatrix}3e^t\\e^{5t}\end{bmatrix}.
$$
Integrating and taking the constants of integration to be zero yields
$$
{\bf
u}=\frac{1}{10}\begin{bmatrix}15e^t\\e^{5t}\end{bmatrix},
$$
so
$$
{\bf y}_p=Y{\bf u}=
\frac{1}{10}\begin{bmatrix}e^{3t}&e^{-t}\\
e^{3t}&-e^{-t}\end{bmatrix}
\begin{bmatrix}15e^t\\e^{5t}\end{bmatrix}
=\frac{1}{5}\begin{bmatrix}8e^{4t}\\7e^{4t}\end{bmatrix}
$$
is a particular solution of  \eqref{eq:10.7.3}.


\ref{item:10.7.1b} From Theorem~\ref{thmtype:10.7.1}, the general solution of
\eqref{eq:10.7.3} is
\begin{equation} \label{eq:10.7.5}
{\bf y}={\bf y}_p+c_1{\bf y}_1+c_2{\bf y}_2=
\frac{1}{5}\begin{bmatrix}8e^{4t}\\7e^{4t}\end{bmatrix}
+c_1\begin{bmatrix}e^{3t}\\e^{3t}\end{bmatrix}
+c_2\begin{bmatrix}e^{-t}\\-e^{-t}\end{bmatrix},
\end{equation}
which can also be written as
$$
{\bf y}=
\frac{1}{5}\begin{bmatrix}8e^{4t}\\7e^{4t}\end{bmatrix}
+\begin{bmatrix}e^{3t}&e^{-t}\\e^{3t}&-e^{-t}\end{bmatrix}{\bf c},
$$
where ${\bf c}$ is an arbitrary constant vector.

Writing \eqref{eq:10.7.5} in terms of
coordinates yields
\begin{eqnarray*}
y_1&=&\frac{8}{5}e^{4t}+c_1e^{3t}+c_2e^{-t}\\
y_2&=&\frac{7}{5}e^{4t}+c_1e^{3t}-c_2e^{-t},
\end{eqnarray*}
so our result is consistent with Example~\ref{example:10.2.1}.
\end{explanation}
\end{example}

If $A$ isn't  a constant matrix, it's usually difficult to find a
fundamental set of solutions for the system ${\bf y}'=A(t){\bf y}$. It
is beyond the scope of this text to discuss methods for doing this.
Therefore, in the following examples and in the exercises involving
systems with variable coefficient matrices we'll provide fundamental
matrices for the complementary systems without explaining how they
were obtained.

\begin{example}\label{example:10.7.2} 
Find a particular solution of
\begin{equation} \label{eq:10.7.6}
{\bf y}'=\begin{bmatrix}2&2e^{-2t}\\2e^{2t}&4\end{bmatrix}{\bf y}+\begin{bmatrix}1\\1\end{bmatrix},
\end{equation}
given that
$$
Y=\begin{bmatrix} e^{4t}&-1\\e^{6t}&e^{2t}\end{bmatrix}
$$
is  a fundamental matrix for the complementary system.

\begin{explanation}
We seek a particular solution
${\bf y}_p=Y{\bf u}$ of \eqref{eq:10.7.6} where
 $Y{\bf u}'={\bf f}$; that is,
$$
\begin{bmatrix} e^{4t}&-1\\e^{6t}&e^{2t}\end{bmatrix}
\begin{bmatrix}u_1'\\u_2'\end{bmatrix}=\begin{bmatrix}1\\1\end{bmatrix}.
$$
The determinant of $Y$ is the Wronskian
$$
\begin{vmatrix}
e^{4t}&-1\\e^{6t}&e^{2t}\end{vmatrix}=2e^{6t}.
$$
By Cramer's rule,
$$
\begin{array}{cccccll}
u_1'&=&\frac{1}{2e^{6t}}\begin{vmatrix}1&-1\\1&e^{2t}
\end{vmatrix}&=&\frac{e^{2t}+1}{2e^{6t}}&=&\frac{e^{-4t}+e^{-6t}}{2}
\\
u_2'&=&\frac{1}{2e^{6t}}\begin{vmatrix}e^{4t}&1\\e^{6t}&1
\end{vmatrix}&=&\frac{e^{4t}-e^{6t}}{2e^{6t}}&=&\frac{e^{-2t}-1}{2}.
\end{array}
$$
Therefore
$$
{\bf
u}'=\frac{1}{2}\begin{bmatrix}e^{-4t}+e^{-6t}\\e^{-2t}-1\end{bmatrix}.
$$
Integrating  and taking the constants of integration to be zero yields
$$
{\bf
u}=-\frac{1}{24}\begin{bmatrix}3e^{-4t}+2e^{-6t}\\6e^{-2t}+12t\end{bmatrix},
$$
so
$$
{\bf y}_p=Y{\bf u}=
-\frac{1}{24}\begin{bmatrix}e^{4t}&-1\\e^{6t}&e^{2t}\end{bmatrix}
\begin{bmatrix}3e^{-4t}+2e^{-6t}\\6e^{-2t}+12t\end{bmatrix}
=\frac{1}{24}\begin{bmatrix}4e^{-2t}+12t-3\\-3e^{2t}(4t+1)-8\end{bmatrix}
$$
is a particular solution of  \eqref{eq:10.7.6}.
\end{explanation}
\end{example}

\begin{example}\label{example:10.7.3}
Find a particular solution of
\begin{equation} \label{eq:10.7.7}
{\bf y}'=-\frac{2}{t^2}\begin{bmatrix}t&-3t^2\\1&-2t\end{bmatrix}{\bf y}
+t^2\begin{bmatrix}1\\1\end{bmatrix},
\end{equation}
given that
$$
Y=\begin{bmatrix}2t&3t^2\\1&2t\end{bmatrix}
$$
is a fundamental matrix for the complementary system on $(-\infty,0)$
and $(0,\infty)$.

\begin{explanation}
We seek a particular solution ${\bf y}_p=Y{\bf u}$ of \eqref{eq:10.7.7}
where $Y{\bf u}'={\bf f}$; that is,
$$
\begin{bmatrix}2t&3t^2\\1&2t\end{bmatrix}\begin{bmatrix}u_1'\\u_2'\end{bmatrix}
=\begin{bmatrix}t^2\\t^2\end{bmatrix}.
$$
The determinant of $Y$  is the Wronskian
$$
\begin{vmatrix}2t&3t^2\\1&2t\end{vmatrix}=t^2.
$$
By Cramer's rule,
$$
\begin{array}{ccccccl}
u_1'&=&\frac{1}{t^2}\begin{vmatrix}t^2&3t^2\\t^2&2t
\end{vmatrix}&=&\frac{2t^3-3t^4}{t^2}&=&2t-3t^2,
\\
u_2'&=&\frac{1}{t^2}\begin{vmatrix}2t&t^2\\1&t^2
\end{vmatrix}&=&\frac{2t^3-t^2}{t^2}&=&2t-1.
\end{array}
$$
Therefore
$$
{\bf u}'=\begin{bmatrix}2t-3t^2\\2t-1\end{bmatrix}.
$$
Integrating  and taking the constants of integration to be zero yields
$$
{\bf u}=\begin{bmatrix}t^2-t^3\\t^2-t
\end{bmatrix},
$$
so
$$
{\bf y}_p=Y{\bf u}=
\begin{bmatrix}2t&3t^2\\1&2t\end{bmatrix}
\begin{bmatrix}t^2-t^3\\t^2-t\end{bmatrix}
=\begin{bmatrix}t^3(t-1)\\t^2(t-1)\end{bmatrix}
$$
is a particular solution of  \eqref{eq:10.7.7}.
\end{explanation}
\end{example}

\begin{example}\label{example:10.7.4} 
\begin{enumerate}
\item\label{item:10.7.4a} % (a)
Find a particular solution of
\begin{equation} \label{eq:10.7.8}
{\bf y}'=\begin{bmatrix}2&-1&-1\\1&0&-1\\1&-1&0\end{bmatrix}{\bf
y}+\begin{bmatrix}e^{t}\\0\\e^{-t}
\end{bmatrix}.
\end{equation}
\item\label{item:10.7.4b} % (b)
Find the general solution of  \eqref{eq:10.7.8}.
\end{enumerate}

\begin{explanation}\ref{item:10.7.4a} 
The complementary system for \eqref{eq:10.7.8} is
\begin{equation} \label{eq:10.7.9}
{\bf y}'=\begin{bmatrix}2&-1&-1\\1&0&-1\\1&-1&0\end{bmatrix}{\bf y}.
\end{equation}
The characteristic polynomial of the coefficient matrix is
$$
\begin{vmatrix}2-\lambda&-1&-1\\1&-\lambda&-1\\1&-1&-\lambda
\end{vmatrix}=
-\lambda(\lambda-1)^2.
$$
Using the method of Section~10.4, we find that
$$
{\bf y}_1=\begin{bmatrix}1\\1\\1\end{bmatrix},\quad
{\bf y}_2=\begin{bmatrix}e^t\\e^t\\0\end{bmatrix},
\quad\mbox{and}\quad
{\bf y}_3=\begin{bmatrix}e^t\\0\\e^t\end{bmatrix}
$$
are linearly independent solutions of \eqref{eq:10.7.9}.
Therefore
$$
Y=\begin{bmatrix}1&e^t&e^t\\1&e^t&0\\1&0&e^t\end{bmatrix}
$$
is a fundamental matrix for  \eqref{eq:10.7.9}.
We seek a particular solution  ${\bf y}_p=Y{\bf u}$ of
\eqref{eq:10.7.8}, where
$Y{\bf u}'={\bf f}$; that is,
$$
\begin{bmatrix}1&e^t&e^t\\1&e^t&0\\1&0&e^t\end{bmatrix}
\begin{bmatrix}u_1'\\u_2'\\u_3'\end{bmatrix}=
\begin{bmatrix}e^t\\0\\e^{-t}\end{bmatrix}.
$$
The determinant of $Y$  is the Wronskian
$$
\begin{vmatrix}1&e^t&e^t\\1&e^t&0\\1&0&e^t\end{vmatrix}
=-e^{2t}.
$$
Thus, by Cramer's rule,
$$
\begin{array}{cccccll}
u_1'&=&-\frac{1}{e^{2t}}\begin{vmatrix}e^t&e^t&e^t\\0&e^t&0\\e^{-t}&0&e^t
\end{vmatrix}&=&-\frac{e^{3t}-e^t}{e^{2t}}&=&e^{-t}-e^t\\ u_2'&=&-\frac{1}{e^{2t}}\begin{vmatrix}1&e^t&e^t\\1&0&0\\1&e^{-t}&e^t
\end{vmatrix}&=&-\frac{1-e^{2t}}{e^{2t}}&=&1-e^{-2t}\\
u_3'&=&-\frac{1}{e^{2t}}\begin{vmatrix}1&e^t&e^t\\1&e^t&0\\1&0&e^{-t}
\end{vmatrix}&=&\frac{e^{2t}}{e^{2t}}&=&1.
\end{array}
$$
 Therefore
$$
{\bf u}'=\begin{bmatrix}e^{-t}-e^t\\1-e^{-2t}\\1\end{bmatrix}.
$$
Integrating and taking the constants of integration to be zero yields
$$
{\bf
u}=\begin{bmatrix}-e^t-e^{-t}\\\frac{1}{2}e^{-2t}+t
\\t\end{bmatrix},
$$
so
$$
\begin{array}{ccccl}
{\bf y}_p=Y{\bf u}&=&
\begin{bmatrix}1&e^t&e^t\\1&e^t&0\\1&0&e^t\end{bmatrix}
\begin{bmatrix}-e^t-e^{-t}\\\frac{1}{2}e^{-2t}+t\\t\end{bmatrix}
&=&\begin{bmatrix}e^t(2t-1)-\frac{e^{-t}}{2}\\
e^t(t-1)-\frac{e^{-t}}{2}\\
e^t(t-1)-e^{-t}
\end{bmatrix}
\end{array}
$$
is a particular solution of  \eqref{eq:10.7.8}.

\ref{item:10.7.4b}
From Theorem~\ref{thmtype:10.7.1} the general solution of
\eqref{eq:10.7.8} is
$$
{\bf y}={\bf y}_p+c_1{\bf y}_1+c_2{\bf y}_2+c_3{\bf y}_3=
\begin{bmatrix}e^t(2t-1)-\frac{e^{-t}}{2}\\
e^t(t-1)-\frac{e^{-t}}{2}\\
e^t(t-1)-e^{-t}\end{bmatrix}+
c_1\begin{bmatrix}1\\1\\1\end{bmatrix}+
c_2\begin{bmatrix}e^t\\e^t\\0\end{bmatrix}
+c_3\begin{bmatrix}e^t\\0\\e^t\end{bmatrix},
$$
which can  be written as
$$
{\bf y}={\bf y}_p+Y{\bf c}=
\begin{bmatrix}e^t(2t-1)-\frac{e^{-t}}{2}\\
e^t(t-1)-\frac{e^{-t}}{2}\\
e^t(t-1)-e^{-t} \end{bmatrix}+
\begin{bmatrix}1&e^t&e^t\\1&e^t&0\\1&0&e^t\end{bmatrix}{\bf c}
$$
where ${\bf c}$ is an arbitrary constant vector.
\end{explanation}
\end{example}

\begin{example}\label{example:10.7.5}
Find a particular solution of
\begin{equation} \label{eq:10.7.10}
{\bf y}'=\frac{1}{2}
\begin{bmatrix}3&e^{-t}&-e^{2t}\\0&6&0\\-e^{-2t}&e^{-3t}&-1\end{bmatrix}
{\bf y}+\begin{bmatrix}1\\e^t\\e^{-t}\end{bmatrix},
\end{equation}
given that
$$
Y=\begin{bmatrix}e^t&0&e^{2t}\\0&e^{3t}&e^{3t}\\e^{-t}&1&0
\end{bmatrix}
$$
is a fundamental matrix for the complementary system.


\begin{explanation}
We seek a particular solution of  \eqref{eq:10.7.10} in the form
${\bf y}_p=Y{\bf u}$, where $Y{\bf u}'={\bf f}$; that is,
$$
\begin{bmatrix}e^t&0&e^{2t}\\0&e^{3t}&e^{3t}\\e^{-t}&1&0
\end{bmatrix}\begin{bmatrix}u_1'\\u_2'\\u_3'\end{bmatrix}=\begin{bmatrix}1\\e^t\\e^{-t}\end{bmatrix}.
$$
The determinant of $Y$  is the Wronskian
$$
\begin{vmatrix}e^t&0&e^{2t}\\0&e^{3t}&e^{3t}\\e^{-t}&1&0
\end{vmatrix}=-2e^{4t}.
$$
By Cramer's rule,
$$
\begin{array}{ccccccl}
u_1'&=&-\frac{1}{2e^{4t}}\begin{vmatrix}1&0&e^{2t}\\e^t&e^{3t}&e^{3t}
\\e^{-t}&1&0
\end{vmatrix}&=&\frac{e^{4t}}{2e^{4t}}&=&\frac{1}{2}
\\
u_2'&=&-\frac{1}{2e^{4t}}\begin{vmatrix}e^t&1&e^{2t}\\0&e^t&e^{3t}
\\e^{-t}&e^{-t}&0
\end{vmatrix}&=&\frac{e^{3t}}{2e^{4t}}&=&\frac{1}{2}e^{-t}\\
u_3'&=&-\frac{1}{2e^{4t}}\begin{vmatrix}e^t&0&1\\0&e^{3t}&e^t
\\e^{-t}&1&e^{-t}
\end{vmatrix}&=&-\frac{e^{3t}-2e^{2t}}{2e^{4t}}&=&\frac{2e^{-2t}-e^{-t}}{2}.
\end{array}
$$
Therefore
$$
{\bf
u}'=\frac{1}{2}\begin{bmatrix}1\\e^{-t}\\2e^{-2t}-e^{-t}\end{bmatrix}.
$$
Integrating  and taking the constants of integration to be zero yields
$$
{\bf u}=\frac{1}{2}\begin{bmatrix}t\\-e^{-t}\\e^{-t}-e^{-2t}
\end{bmatrix},
$$
so
$$
{\bf y}_p=Y{\bf u}=
\frac{1}{2}\begin{bmatrix}e^t&0&e^{2t}\\0&e^{3t}&e^{3t}\\e^{-t}&1&0
\end{bmatrix}
\begin{bmatrix}t\\-e^{-t}\\e^{-t}-e^{-2t}
\end{bmatrix}=\frac{1}{2}\begin{bmatrix}e^t(t+1)-1\\
-e^t\\e^{-t}(t-1)\end{bmatrix}
$$
is a particular solution of  \eqref{eq:10.7.10}.
\end{explanation}
\end{example}




\section*{Text Source}
Trench, William F., "Elementary Differential Equations" (2013). Faculty Authored and Edited Books \& CDs. 8. (CC-BY-NC-SA)

\href{https://digitalcommons.trinity.edu/mono/8/}{https://digitalcommons.trinity.edu/mono/8/}


\end{document}