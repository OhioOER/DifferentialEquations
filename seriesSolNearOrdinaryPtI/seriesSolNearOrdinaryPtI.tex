\documentclass{ximera}

%% You can put user macros here
%% However, you cannot make new environments

\listfiles

%\graphicspath{{./}{firstExample/}{secondExample/}}
\graphicspath{{./}
{aboutDiffEq/}
{applicationsLeadingToDiffEq/}
{applicationsToCurves/}
{autonomousSecondOrder/}
{basicConcepts/}
{bernoulli/}
{constCoeffHomSysI/}
{constCoeffHomSysII/}
{constCoeffHomSysIII/}
{constantCoeffWithImpulses/}
{constantCoefficientHomogeneousEquations/}
{convolution/}
{coolingActivity/}
{directionFields/}
{drainingTank/}
{epidemicActivity/}
{eulersMethod/}
{exactEquations/}
{existUniqueNonlinear/}
{frobeniusI/}
{frobeniusII/}
{frobeniusIII/}
{global.css/}
{growthDecay/}
{heatingCoolingActivity/}
{higherOrderConstCoeff/}
{homogeneousLinearEquations/}
{homogeneousLinearSys/}
{improvedEuler/}
{integratingFactors/}
{interactExperiment/}
{introToLaplace/}
{introToSystems/}
{inverseLaplace/}
{ivpLaplace/}
{laplaceTable/}
{lawOfCooling/}
{linSysOfDiffEqs/}
{linearFirstOrderDiffEq/}
{linearHigherOrder/}
{mixingProblems/}
{motionUnderCentralForce/}
{nonHomogeneousLinear/}
{nonlinearToSeparable/}
{odesInSage/}
{piecewiseContForcingFn/}
{population/}
{reductionOfOrder/}
{regularSingularPts/}
{reviewOfPowerSeries/}
{rlcCircuit/}
{rungeKutta/}
{secondLawOfMotion/}
{separableEquations/}
{seriesSolNearOrdinaryPtI/}
{seriesSolNearOrdinaryPtII/}
{simplePendulum/}
{springActivity/}
{springProblemsI/}
{springProblemsII/}
{undCoeffHigherOrderEqs/}
{undeterminedCoeff/}
{undeterminedCoeff2/}
{unitStepFunction/}
{varParHigherOrder/}
{varParamNonHomLinSys/}
{variationOfParameters/}
}


\usepackage{tikz}
\usepackage{tkz-euclide}
\usepackage{tikz-3dplot}
\usepackage{tikz-cd}
\usetikzlibrary{shapes.geometric}
\usetikzlibrary{arrows}
\usetikzlibrary{decorations.pathmorphing,patterns}
\usetikzlibrary{backgrounds} % added by Felipe
% \usetkzobj{all}   % NOT ALLOWED IN RECENT TeX's ...
\pgfplotsset{compat=1.13} % prevents compile error.

\renewcommand{\vec}[1]{\mathbf{#1}}
\newcommand{\RR}{\mathbb{R}}
\newcommand{\dfn}{\textit}
\newcommand{\dotp}{\cdot}
\newcommand{\id}{\text{id}}
\newcommand\norm[1]{\left\lVert#1\right\rVert}
\newcommand{\dst}{\displaystyle}
 
\newtheorem{general}{Generalization}
\newtheorem{initprob}{Exploration Problem}

\tikzstyle geometryDiagrams=[ultra thick,color=blue!50!black]

\usepackage{mathtools}

\title{Series Solutions Near an Ordinary Point I}%\label{Module 7-ADEF}


\begin{document}

\begin{abstract}
We consider the utilization of power series to determine solutions to certain differential equations.
\end{abstract}

\maketitle

\section*{Series Solutions Near an Ordinary Point I}

Many physical applications give rise to second order homogeneous
linear  differential equations of the form
\begin{equation}\label{eq:7.2.1}
P_0(x)y''+P_1(x)y'+P_2(x)y=0,
\end{equation}
where $P_0$, $P_1$, and $P_2$ are polynomials. Usually the solutions
of these equations can't be expressed in terms of familiar elementary
functions. Therefore we'll consider the problem of representing
solutions of \eqref{eq:7.2.1} with series.

We assume throughout that $P_0$, $P_1$ and $P_2$ have no common
factors. Then we say that $x_0$ is an \textit{ordinary point} of
\eqref{eq:7.2.1} if $P_0(x_0)\neq0$, or a \textit{singular point} if
$P_0(x_0)=0$. For  Legendre's equation,
\begin{equation}\label{eq:7.2.2}
(1-x^2)y''-2xy'+\alpha(\alpha+1)y=0,
\end{equation}
$x_0=1$ and $x_0=-1$ are singular points and all other points are
ordinary points. For  Bessel's~ equation,
$$
x^2y''+xy'+(x^2-\nu^2)y=0,
$$
$x_0=0$ is a singular point and all other points are ordinary points.
If $P_0$ is a nonzero constant as in  Airy's
equation,
\begin{equation}\label{eq:7.2.3}
y''-xy=0,
\end{equation}
then every point is an ordinary point.

Since polynomials are continuous everywhere, $P_1/P_0$ and $P_2/P_0$
are continuous at any point $x_0$ that isn't  a zero of $P_0$.
Therefore, if $x_0$ is an ordinary point of \eqref{eq:7.2.1} and $a_0$ and
$a_1$ are arbitrary real numbers, then the initial value problem
\begin{equation}\label{eq:7.2.4}
P_0(x)y''+P_1(x)y'+P_2(x)y=0, \quad  y(x_0)=a_0,\quad y'(x_0)=a_1
\end{equation}
has a unique solution on the largest open interval that contains $x_0$
and does not contain any zeros of $P_0$. To see this, we rewrite the
differential equation in \eqref{eq:7.2.4} as
$$
y''+\frac{P_1(x)}{P_0(x)}y'+\frac{P_2(x)}{P_0(x)}y=0
$$
and apply Theorem~\ref{thmtype:5.1.1} with $p=P_1/P_0$ and $q=P_2/P_0$.
In this section and the next we  consider the problem of
representing  solutions of \eqref{eq:7.2.1}  by power series that
converge  for values of $x$  near an ordinary point $x_0$.

We state the next theorem without proof.

\begin{theorem}\label{thmtype:7.2.1}
Suppose $P_0$, $P_1$, and $P_2$ are polynomials with no common
factor and $P_0$ isn't identically zero. Let $x_0$ be a point such
that $P_0(x_0)\neq0$, and let $\rho$ be the distance from $x_0$ to the
nearest zero of $P_0$ in the complex plane. (If $P_0$ is constant,
then $\rho=\infty$.) Then every solution of
\begin{equation}\label{eq:7.2.5}
P_0(x)y''+P_1(x)y'+P_2(x)y=0
\end{equation}
can be represented by a power series
 \begin{equation}\label{eq:7.2.6}
y=\sum_{n=0}^\infty a_n(x-x_0)^n
\end{equation}
that converges at least on the open interval $(x_0-\rho,x_0+\rho)$.
(If $P_0$ is nonconstant, so that $\rho$ is necessarily finite,
then the open interval of convergence of $\eqref{eq:7.2.6}$ may be larger
than $(x_0-\rho,x_0+\rho)$. If $P_0$ is constant then $\rho=\infty$
and $(x_0-\rho,x_0+\rho)=(-\infty,\infty)$.)
\end{theorem}

We call \eqref{eq:7.2.6} a \textit{power series solution in $x-x_0$}  of
\eqref{eq:7.2.5}. We'll now develop a method for finding power series
solutions of \eqref{eq:7.2.5}. For this purpose we write \eqref{eq:7.2.5} as
$Ly=0$, where
\begin{equation}\label{eq:7.2.7}
Ly=P_0y''+P_1y'+P_2y.
\end{equation}

Theorem~\ref{thmtype:7.2.1} implies that every solution of $Ly=0$ on
$(x_0-\rho,x_0+\rho)$ can be written as
$$
y=\sum_{n=0}^\infty a_n(x-x_0)^n.
$$
Setting $x=x_0$ in this series and in the series
$$
y'=\sum_{n=1}^\infty na_n(x-x_0)^{n-1}
$$
shows that $y(x_0)=a_0$ and $y'(x_0)=a_1$. Since every initial value
problem \eqref{eq:7.2.4} has a unique solution, this means that $a_0$ and
$a_1$ can be chosen arbitrarily, and  $a_2, a_3, \dots$ are
uniquely determined by them.

To find $a_2, a_3, \dots$, we write $P_0$, $P_1$, and $P_2$ in powers of
$x-x_0$, substitute
$$
y=\sum^\infty_{n=0}a_n(x-x_0)^n,
$$
$$
y'=\sum^\infty_{n=1}na_n(x-x_0)^{n-1},
$$
$$
y''=\sum^\infty_{n=2}n(n-1)a_n(x-x_0)^{n-2}
$$
into \eqref{eq:7.2.7}, and collect the coefficients of like powers of
$x-x_0$. This yields
\begin{equation}\label{eq:7.2.8}
Ly=\sum^\infty_{n=0}b_n(x-x_0)^n,
\end{equation}
where $\{b_0, b_1, \dots, b_n, \dots\}$ are expressed in terms of
$\{a_0, a_1, \dots,a_n, \dots\}$ and the coefficients of $P_0$, $P_1$,
and $P_2$, written in powers of $x-x_0$. Since \eqref{eq:7.2.8} and
the first part %\part{a} 
of Theorem~\ref{thmtype:7.1.6} imply that $Ly=0$ if and only if
$b_n=0$ for $n\geq0$,  all power series solutions in
$x-x_0$ of $Ly=0$ can be obtained by choosing $a_0$ and $a_1$
arbitrarily and computing $a_2, a_3, \dots$, successively so that
$b_n=0$
for $n\geq0$. For simplicity, we call the power series obtained  this
way \textit{the power series in $x-x_0$ for the general solution} of
$Ly=0$, without explicitly identifying the open interval of
convergence of the series.

\begin{example}\label{example:7.2.1}
Let $x_0$ be an arbitrary real number. Find the power series in
$x-x_0$ for the general solution of
\begin{equation}\label{eq:7.2.9}
 y''+ y=0.
\end{equation}
\begin{explanation}
Here
$$
Ly=y''+y.
$$
If
$$
y=\sum_{n=0}^\infty a_n(x-x_0)^n,
$$
then
$$
y''=\sum_{n=2}^\infty n(n-1)a_n(x-x_0)^{n-2},
$$
so
$$
Ly=\sum_{n=2}^\infty n(n-1)a_n(x-x_0)^{n-2}+\sum_{n=0}^\infty a_n(x-x_0)^n.
$$
To collect coefficients of like powers of $x-x_0$, we shift the summation
index in the first sum. This yields
$$
Ly=\sum^\infty_{n=0}(n+2)(n+1)a_{n+2}(x-x_0)^n +
 \sum^\infty_{n=0}a_n(x-x_0)^n
=\sum^\infty_{n=0}b_n(x-x_0)^n,
$$
with
$$
b_n=(n+2)(n+1)a_{n+2}+a_n.
$$
Therefore $Ly=0$ if and only if
\begin{equation}\label{eq:7.2.10}
a_{n+2}=\frac{-a_n}{(n+2)(n+1)},\quad n\geq0,
\end{equation}
where $a_0$ and $a_1$ are arbitrary. Since the indices on the left and
right sides of \eqref{eq:7.2.10} differ by two, we write \eqref{eq:7.2.10}
separately for $n$ even $(n=2m)$ and $n$ odd $(n=2m+1)$. This yields
\begin{eqnarray}
a_{2m+2}&=&\frac{-a_{2m}}{(2m+2)(2m+1)},\quad m\geq0,
\label{eq:7.2.11} \\%dummy \eqref{eq:7.2.14}
\noalign{\mbox{and}}\nonumber\\
a_{2m+3}&=&\frac{-a_{2m+1}}{(2m+3)(2m+2)},\quad  m\geq0.
\label{eq:7.2.12}
\end{eqnarray}
Computing the coefficients of the even powers of $x-x_0$ from
\eqref{eq:7.2.11} yields
\begin{eqnarray*}
a_2&=&-\frac{a_0}{2\cdot1}\\
a_4&=&-\frac{a_2}{4\cdot3}=-\frac{1}{4\cdot3}
 \left(-\frac{a_0}{2\cdot1}\right)=
 \frac{a_0}{4\cdot3\cdot2\cdot1}, \\
a_6&=&-\frac{a_4}{6\cdot5}=-\frac{1}{6\cdot5}
 \left(\frac{a_0}{4\cdot3\cdot2\cdot1}\right)
=-\frac{a_0}{6\cdot5\cdot4\cdot3\cdot
 2\cdot1},
\end{eqnarray*}
and, in general,
\begin{equation}\label{eq:7.2.13}
a_{2m}=(-1)^m \frac{a_0}{(2m)!}   ,\quad  m\geq0.
\end{equation}
Computing the coefficients of the odd powers of $x-x_0$ from \eqref{eq:7.2.12}
yields
\begin{eqnarray*}
a_3&=&-\frac{a_1}{3\cdot2}\\
a_5&=&-\frac{a_3}{5\cdot4}=-\frac{1}{5\cdot4}
 \left(-\frac{a_1}{3\cdot2}\right)=
 \frac{a_1}{5\cdot4\cdot3\cdot2}, \\
a_7&=&-\frac{a_5}{7\cdot6}=-\frac{1}{7\cdot6}
 \left({a_1}{5\cdot4\cdot3\cdot2}\right)
=-{a_1}{7\cdot6\cdot5\cdot4\cdot
 3\cdot2},
\end{eqnarray*}
and, in general,
\begin{equation}\label{eq:7.2.14}
a_{2m+1}=\frac{(-1)^ma_1}{(2m+1)!}\quad m\geq0.
\end{equation}
Thus, the general solution of \eqref{eq:7.2.9} can be written as
$$
y=\sum_{m=0}^\infty a_{2m}(x-x_0)^{2m}+\sum_{m=0}^\infty a_{2m+1}(x-x_0)^{2m+1},
$$
or, from \eqref{eq:7.2.13} and \eqref{eq:7.2.14}, as
\begin{equation}\label{eq:7.2.15}
y=a_0\sum_{m=0}^\infty(-1)^m\frac{(x-x_0)^{2m}}{(2m)!}
+a_1\sum_{m=0}^\infty(-1)^m\frac{(x-x_0)^{2m+1}}{(2m+1)!}.
\end{equation}
If we recall from calculus that
$$
\sum_{m=0}^\infty(-1)^m\frac{(x-x_0)^{2m}}{(2m)!}=\cos(x-x_0)
\quad\mbox{and}\quad
\sum_{m=0}^\infty(-1)^m\frac{(x-x_0)^{2m+1}}{(2m+1)!}=\sin(x-x_0),
$$
then \eqref{eq:7.2.15} becomes
$$
y=a_0\cos(x-x_0)+a_1\sin(x-x_0),
$$
which should look familiar.
\end{explanation}
\end{example}

Equations like \eqref{eq:7.2.10}, \eqref{eq:7.2.11}, and \eqref{eq:7.2.12}, which
define a given coefficient in the sequence $\{a_n\}$ in terms of one
or more coefficients with lesser indices are called \textit{recurrence relations}. When we use a recurrence relation to compute terms of a sequence we're computing \textit{recursively}.

In  the remainder of this  section we  consider the problem of
finding power series solutions in $x-x_0$  for equations of the form
\begin{equation}\label{eq:7.2.16}
\left(1+\alpha(x-x_0)^2\right)y''+\beta(x-x_0) y'+\gamma y=0.
\end{equation}
Many important equations that arise in applications are of this
form with $x_0=0$, including  Legendre's equation \eqref{eq:7.2.2},
Airy's equation \eqref{eq:7.2.3},
\href{http://www-history.mcs.st-and.ac.uk/Mathematicians/Chebyshev.html}{Chebyshev's  equation},
$$
(1-x^2)y''-xy'+\alpha^2 y=0,
$$
and
\href{http://www-history.mcs.st-and.ac.uk/Mathematicians/Hermite.html}{Hermite's equation},
$$
y''-2xy'+2\alpha y=0.
$$
Since
$$
P_0(x)=1+\alpha(x-x_0)^2
$$
in \eqref{eq:7.2.16}, the point $x_0$ is an ordinary point of
\eqref{eq:7.2.16}, and Theorem~\ref{thmtype:7.2.1} implies that the solutions
of \eqref{eq:7.2.16} can be written as power series in $x-x_0$ that
converge on the interval $(x_0-1/\sqrt|\alpha|,x_0+1/\sqrt|\alpha|)$
if $\alpha\neq0$, or on $(-\infty,\infty)$ if $\alpha=0$. We'll see
that the coefficients in these power series can be obtained by methods
similar to the one used in Example~\ref{example:7.2.1}.

To simplify
finding the coefficients, we introduce some notation for
products:
$$
\prod^s_{j=r}b_j=b_rb_{r+1}\cdots b_s\quad \mbox{if}
\quad s\geq r.
$$
Thus,
$$
\prod^7_{j=2}b_j=b_2b_3b_4b_5b_6b_7,
$$
$$
\prod^4_{j=0}(2j+1)=(1)(3)(5)(7)(9)=945,
$$
and
$$
\prod^2_{j=2}j^2=2^2=4.
$$
We define
$$
\prod^s_{j=r}b_j=1\quad \mbox{if}\quad s < r,
$$
no matter what the form of $b_j$.

\begin{example}\label{example:7.2.2}
Find the power series in $x$ for the general solution of
\begin{equation}\label{eq:7.2.17}
 (1+2x^2)y''+6xy'+2y=0.
\end{equation}
\begin{explanation}
Here
$$
Ly=(1+2x^2)y''+6xy'+2y.
$$
If
$$
y=\sum_{n=0}^\infty a_nx^n
$$
then
$$
y'=\sum_{n=1}^\infty na_nx^{n-1}\quad\mbox{ and }\quad
y''=\sum_{n=2}^\infty n(n-1)a_nx^{n-2},
$$
so
\begin{eqnarray*}
Ly&=&(1+2x^2) \sum^\infty_{n=2}n(n-1)a_nx^{n-2}+ 6x
 \sum^\infty_{n=1}na_nx^{n-1}
 +2 \sum^\infty_{n=0}a_nx^n\\
&=&\sum_{n=2}^\infty n(n-1)a_nx^{n-2}+\sum_{n=0}^\infty
\left[2n(n-1)+6n+2\right]a_nx^n\\
&=&\sum_{n=2}^\infty n(n-1)a_nx^{n-2}+2\sum_{n=0}^\infty(n+1)^2a_nx^n.
\end{eqnarray*}
To collect coefficients of $x^n$,  we shift the summation index
in the first sum.  This yields
$$
Ly=\sum_{n=0}^\infty(n+2)(n+1)a_{n+2}x^n+2\sum_{n=0}^\infty(n+1)^2a_nx^n
=\sum_{n=0}^\infty b_nx^n,
$$
with
$$
b_n=(n+2)(n+1)a_{n+2}+2(n+1)^2a_n,\quad n\geq0.
$$
To obtain solutions of \eqref{eq:7.2.17}, we set $b_n=0$ for $n\geq0$. This
is equivalent to the recurrence relation
 \begin{equation}\label{eq:7.2.18}
a_{n+2}=-2\frac{n+1}{n+2}a_n,\quad n\geq0.
\end{equation}
Since the
indices on the left and right differ by two, we write \eqref{eq:7.2.18}
separately for $n=2m$ and $n=2m+1$, as in Example~\ref{example:7.2.1}.
This yields
\begin{eqnarray}
a_{2m+2}&=&-2 \frac{2m+1}{2m+2}a_{2m}=-\frac{2m+1}{m+1}a_{2m},\quad m
\geq0,\label{eq:7.2.19}\\
\noalign{\mbox{and}}\nonumber\\
a_{2m+3}&=&-2\frac{2m+2}{2m+3}a_{2m+1}=-4\frac{m+1}{2m+3}a_{2m+1},\quad
m\geq0. \label{eq:7.2.20}
\end{eqnarray}
Computing the coefficients of even powers of $x$ from \eqref{eq:7.2.19}
yields
\begin{eqnarray*}
a_2&=&-\frac{1}{1}a_0,\\
a_4&=&-\frac{3}{2}a_2=\left(-\frac{3}{2}\right)\left(-\frac{1}{1}\right)a_0
=\frac{1\cdot3}{1\cdot2}a_0,
\\ a_6&=&-\frac{5}{3}a_4=
-\frac{5}{3}\left(\frac{1\cdot3}{1\cdot2}\right)a_0
=-\frac{1\cdot3\cdot5}{1\cdot2\cdot3}a_0, \\
a_8&=&-\frac{7}{4}a_6=-{7}{4}
\left(-\frac{1\cdot3\cdot5}{1\cdot2\cdot3}\right)a_0=
\frac{1\cdot3\cdot5\cdot7}{1\cdot2\cdot3\cdot4}a_0.\\
\end{eqnarray*}
In general,
\begin{equation}\label{eq:7.2.21}
a_{2m}=(-1)^m\frac{\prod_{j=1}^m(2j-1)}{m!}a_0,\quad m\geq0.
\end{equation}
(Note that \eqref{eq:7.2.21} is correct for $m=0$  because we
defined $\prod_{j=1}^0b_j=1$  for any~$b_j$.)

Computing the coefficients of odd powers of $x$ from \eqref{eq:7.2.20}
yields
\begin{eqnarray*}
a_3&=&-4\,\frac{1}{3}a_1, \\
a_5&=&-4\,\frac{2}{5}a_3=-4\,\frac{2}{5}\left(-4\frac{1}{3}\right)a_1
=4^2\frac{1\cdot2}{3\cdot5}a_1,
\\ a_7&=&-4\,\frac{3}{7}a_5=-4\,\frac{3}{7}\left(
4^2\frac{1\cdot2}{3\cdot5}\right)a_1=
-4^3\frac{1\cdot2\cdot3}{3\cdot5\cdot7}a_1,\\
a_9&=&-4\, \frac{4}{9}a_7=-4\, {4}{9}\left(
4^3\frac{1\cdot2\cdot3}{3\cdot5\cdot7}\right)a_1=
4^4\frac{1\cdot2\cdot3\cdot4}{3\cdot5\cdot7\cdot9}a_1.
\end{eqnarray*}
In general,
\begin{equation}\label{eq:7.2.22}
a_{2m+1}=\frac{(-1)^m4^m m!}{\prod_{j=1}^m(2j+1)}a_1,\quad m\geq0.
\end{equation}
From \eqref{eq:7.2.21} and \eqref{eq:7.2.22},
$$
y=a_0
 \sum^\infty_{m=0}(-1)^m \frac{\prod_{j=1}^m(2j-1)}{m!}x^{2m}
+a_1 \sum^\infty_{m=0}(-1)^m \frac{4^mm!}{\prod_{j=1}^m(2j+1)}
 x^{2m+1}.
$$
is the power series in $x$ for the general solution of \eqref{eq:7.2.17}.
Since $P_0(x)=1+2x^2$ has no real zeros, Theorem~\ref{thmtype:5.1.1} implies
that every solution of \eqref{eq:7.2.17} is defined on $(-\infty,\infty)$.
However, since $P_0(\pm i/\sqrt2)=0$, Theorem~\ref{thmtype:7.2.1} implies
only that the power series converges in $(-1/\sqrt2,1/\sqrt2)$ for any
choice of $a_0$ and $a_1$.
\end{explanation}
\end{example}

The results in Examples~\ref{example:7.2.1} and \ref{example:7.2.2} are
consequences of the following general theorem.

\begin{theorem}\label{thmtype:7.2.2}
The coefficients $\{a_n\}$ in any solution
$y=\sum_{n=0}^\infty a_n(x-x_0)^n$ of
\begin{equation}\label{eq:7.2.23}
\left(1+\alpha(x-x_0)^2\right)y''+\beta(x-x_0) y'+\gamma y=0
\end{equation}
satisfy the recurrence relation
\begin{equation}\label{eq:7.2.24}
a_{n+2}=-\frac{p(n)}{(n+2)(n+1)}a_n,\quad  n\geq0,
\end{equation}
where
\begin{equation}\label{eq:7.2.25}
p(n)=\alpha n(n-1) +\beta n+\gamma.
\end{equation}
Moreover$,$ the coefficients of the even and odd powers of $x-x_0$ can
be computed separately as
\begin{eqnarray}
a_{2m+2}&=&-\frac{p(2m)}{(2m+2)(2m+1)}a_{2m},\quad m\geq0\label{eq:7.2.26}\\
\noalign{\mbox{and}}\nonumber\\
a_{2m+3}&=&-\frac{p(2m+1)}{(2m+3)(2m+2)}a_{2m+1},\quad m\geq0, \label{eq:7.2.27}
\end{eqnarray}
where $a_0$ and $a_1$ are arbitrary.
\end{theorem}

\begin{proof}
Here
$$
Ly=\left(1+\alpha(x-x_0\right)^2)y''+\beta(x-x_0) y'+\gamma y.
$$
If
$$
y=\sum_{n=0}^\infty a_n(x-x_0)^n,
$$
then
$$
y'=\sum_{n=1}^\infty na_n(x-x_0)^{n-1}
\quad\mbox{ and }\quad y''=\sum_{n=2}^\infty n(n-1)a_n(x-x_0)^{n-2}.
$$
Hence,
$$
\begin{array}{ccl}
Ly&=&\sum_{n=2}^\infty n(n-1)a_n(x-x_0)^{n-2}+
\sum_{n=0}^\infty \left[\alpha
n(n-1)
+\beta n+\gamma\right]a_n(x-x_0)^n\\
&=&\sum_{n=2}^\infty n(n-1)a_n(x-x_0)^{n-2}+\sum_{n=0}^\infty
p(n)a_n(x-x_0)^n,
\end{array}
$$
from \eqref{eq:7.2.25}.  To collect coefficients of powers of $x-x_0$,
we shift the summation index in the first sum. This yields
$$
Ly=\sum_{n=0}^\infty \left[(n+2)(n+1)a_{n+2}+p(n)a_n\right](x-x_0)^n.
$$
Thus, $Ly=0$ if and only if
$$
(n+2)(n+1)a_{n+2}+p(n)a_n=0,\quad n\geq0,
$$
which is equivalent to \eqref{eq:7.2.24}. Writing \eqref{eq:7.2.24} separately
for the cases where $n=2m$ and $n=2m+1$ yields \eqref{eq:7.2.26} and
\eqref{eq:7.2.27}.
\end{proof}

\begin{example}\label{example:7.2.3}
Find the power series in $x-1$  for the general
solution of
\begin{equation}\label{eq:7.2.28}
(2+4x-2x^2)y''-12(x-1)y'-12y=0.
\end{equation}
\begin{explanation}
We must first write the coefficient $P_0(x)=2+4x-x^2$ in powers of
$x-1$. To do this,
 we write $x=(x-1)+1$ in $P_0(x)$ and then expand
the terms, collecting powers of $x-1$;   thus,
\begin{eqnarray*}
2+4x-2x^2&=&2+4[(x-1)+1]-2[(x-1)+1]^2\\
&=&4-2(x-1)^2.
\end{eqnarray*}
  Therefore we can rewrite \eqref{eq:7.2.28} as
$$
\left(4-2(x-1)^2\right)y''-12(x-1)y'-12y=0,
$$
or, equivalently,
$$
\left(1-\frac{1}{2}(x-1)^2\right)y''-3(x-1)y'-3y=0.
$$
This is of the form \eqref{eq:7.2.23} with  $\alpha=-1/2$, $\beta=-3$, and
$\gamma=-3$. Therefore, from \eqref{eq:7.2.25}
$$
p(n)=-\frac{n(n-1)}{2}-3n-3=-\frac{(n+2)(n+3)}{2}.
$$
Hence, Theorem~\ref{thmtype:7.2.2} implies that
\begin{eqnarray*}
a_{2m+2}&=&-\frac{p(2m)}{(2m+2)(2m+1)}a_{2m}\\&=&
\frac{(2m+2)(2m+3)}{2(2m+2)(2m+1)}
a_{2m}=\frac{2m+3}{2(2m+1)}a_{2m},\quad m\geq0\\
\noalign{\mbox{and}}\\
a_{2m+3}&=&-\frac{p(2m+1)}{(2m+3)(2m+2)}a_{2m+1}\\&=&
\frac{(2m+3)(2m+4)}{2
(2m+3)(2m+2)}a_{2m+1}=\frac{m+2}{2(m+1)}a_{2m+1},\quad m\geq0.
\end{eqnarray*}
We leave it to you to show that
$$
a_{2m}=\frac{2m+1}{2^m}a_0\quad\mbox{and}\quad
a_{2m+1}=\frac{m+1}{2^m}a_1,\quad m\geq0,
$$
which implies that the power series in $x-1$ for the general solution
of
\eqref{eq:7.2.28} is
$$
y=a_0\sum_{m=0}^\infty\frac{2m+1}{2^m}(x-1)^{2m}+a_1\sum_{m=0}^\infty
\frac{m+1}{2^m}(x-1)^{2m+1}.
$$
\end{explanation}
\end{example}

In the examples considered so far we were able to obtain closed
formulas for coefficients in the power series solutions. In some cases
this is impossible, and we must settle for computing a finite number
of terms in the series. The next example illustrates this with an
initial value problem.

\begin{example}\label{example:7.2.4}
Compute  $a_0, a_1, \dots, a_7$  in the series solution
$y=\sum_{n=0}^\infty a_nx^n$ of the initial value problem
\begin{equation}\label{eq:7.2.29}
(1+2x^2)y''+10xy'+8y=0,\quad y(0)=2,\quad y'(0)=-3.
\end{equation}
\begin{explanation}
Since $\alpha=2$, $\beta=10$, and $\gamma=8$ in \eqref{eq:7.2.29},
$$
p(n)=2n(n-1)+10n+8=2(n+2)^2.
$$
 Therefore
$$
a_{n+2}=-2\frac{(n+2)^2}{(n+2)(n+1)}a_n=-2\frac{n+2}{n+1}a_n,\quad n\geq0.
$$
Writing this equation separately for $n=2m$  and $n=2m+1$ yields
\begin{eqnarray}
a_{2m+2}&=&-2\frac{(2m+2)}{2m+1}a_{2m}=-4\frac{m+1}{2m+1}a_{2m},\quad m\geq
0\label{eq:7.2.30}\\
\noalign{\mbox{and}}\nonumber\\
a_{2m+3}&=&-2\frac{2m+3}{2m+2}a_{2m+1}=-\frac{2m+3}{m+1}a_{2m+1},\quad m\geq0.
\label{eq:7.2.31}
\end{eqnarray}
Starting with  $a_0=y(0)=2$, we compute $a_2, a_4$, and $a_6$ from
\eqref{eq:7.2.30}:
\begin{eqnarray*}
a_2&=&-4\,\frac{1}{1}2=-8,\\
a_4&=&-4\,\frac{2}{3}(-8)=\frac{64}{3},\\
a_6&=&-4\,\frac{3}{5}\left(\frac{64}{3}\right)=-\frac{256}{5}.
\end{eqnarray*}
Starting with $a_1=y'(0)=-3$, we compute $a_3,a_5$ and $a_7$ from
\eqref{eq:7.2.31}:
\begin{eqnarray*}
a_3&=&-\frac{3}{1}(-3)=9,\\
a_5&=&-\frac{5}{2}9=-\frac{45}{2},\\
a_7&=&-\frac{7}{3}\left(-\frac{45}{2}\right)=\frac{105}{2}.
\end{eqnarray*}
Therefore the solution of \eqref{eq:7.2.29} is
$$
y=2-3x-8x^2+9x^3+\frac{64}{3}x^4-\frac{45}{2}x^5-\frac{256}{5}
x^6+\frac{105}{2}x^7+\cdots .
$$
\end{explanation}
\end{example}

%\technology

\subsection*{A Note on Technology}

Computing coefficients recursively as in Example~\ref{example:7.2.4} is
tedious. We recommend that you do this kind of computation by writing
a short program to implement the appropriate recurrence relation on a
calculator or computer. You may wish to do this in
verifying examples and doing exercises. %(identified by the symbol \Cex) in this chapter that call for numerical computation of the
%coefficients in series solutions. We obtained the answers to these
%exercises by using software that can produce answers in the form of
%rational numbers. However, it's perfectly acceptable - and more
%practical - to get your answers in decimal form. You can always check
%them by converting our fractions to decimals.

If you're interested in actually using series to compute numerical
approximations to solutions of a differential equation, then whether
or not there's a simple closed form for the coefficients is
essentially irrelevant. For computational purposes it's usually
more efficient to start with the given coefficients $a_0=y(x_0)$ and
$a_1=y'(x_0)$, compute $a_2, \dots, a_N$ recursively,
 and then compute approximate values of the solution from the
Taylor polynomial
$$
T_N(x)=\sum_{n=0}^Na_n(x-x_0)^n.
$$
The trick is to decide how to choose $N$ so the approximation
$y(x)\approx T_N(x)$ is sufficiently accurate on the subinterval of
the interval of convergence that you're interested in. In the
computational exercises in this  and the next two sections,
you will often be asked to obtain the solution of a given problem by
numerical integration with  software  of your choice (see
the end of Module \ref{Module 15-1}%Section~10.1 
for a brief discussion of one such method),
and to compare the solution obtained in this way with the
approximations obtained with $T_N$ for various values of $N$. This is
a typical textbook kind of exercise, designed to give you insight into
how the accuracy of the approximation $y(x)\approx T_N(x)$ behaves as
a function of $N$ and the interval that you're working on. In real
life, you would choose one or the other of the two methods (numerical
integration or series solution). If you choose the method of series
solution, then a practical procedure for determining a suitable value
of $N$ is to continue increasing $N$ until the maximum of
$|T_N-T_{N-1}|$ on the interval of interest is within the margin of
error that you're willing to accept.

In doing computational problems  that call for
numerical solution of differential equations you should
choose the most accurate numerical integration procedure your
software supports, and experiment with the step size until
you're confident that the numerical results are sufficiently
accurate for the problem at hand.




\section*{Text Source}
Trench, William F., "Elementary Differential Equations" (2013). Faculty Authored and Edited Books \& CDs. 8. (CC-BY-NC-SA)

\href{https://digitalcommons.trinity.edu/mono/8/}{https://digitalcommons.trinity.edu/mono/8/}


\end{document}