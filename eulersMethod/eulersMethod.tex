\documentclass{ximera}

%% You can put user macros here
%% However, you cannot make new environments

\listfiles

%\graphicspath{{./}{firstExample/}{secondExample/}}
\graphicspath{{./}
{aboutDiffEq/}
{applicationsLeadingToDiffEq/}
{applicationsToCurves/}
{autonomousSecondOrder/}
{basicConcepts/}
{bernoulli/}
{constCoeffHomSysI/}
{constCoeffHomSysII/}
{constCoeffHomSysIII/}
{constantCoeffWithImpulses/}
{constantCoefficientHomogeneousEquations/}
{convolution/}
{coolingActivity/}
{directionFields/}
{drainingTank/}
{epidemicActivity/}
{eulersMethod/}
{exactEquations/}
{existUniqueNonlinear/}
{frobeniusI/}
{frobeniusII/}
{frobeniusIII/}
{global.css/}
{growthDecay/}
{heatingCoolingActivity/}
{higherOrderConstCoeff/}
{homogeneousLinearEquations/}
{homogeneousLinearSys/}
{improvedEuler/}
{integratingFactors/}
{interactExperiment/}
{introToLaplace/}
{introToSystems/}
{inverseLaplace/}
{ivpLaplace/}
{laplaceTable/}
{lawOfCooling/}
{linSysOfDiffEqs/}
{linearFirstOrderDiffEq/}
{linearHigherOrder/}
{mixingProblems/}
{motionUnderCentralForce/}
{nonHomogeneousLinear/}
{nonlinearToSeparable/}
{odesInSage/}
{piecewiseContForcingFn/}
{population/}
{reductionOfOrder/}
{regularSingularPts/}
{reviewOfPowerSeries/}
{rlcCircuit/}
{rungeKutta/}
{secondLawOfMotion/}
{separableEquations/}
{seriesSolNearOrdinaryPtI/}
{seriesSolNearOrdinaryPtII/}
{simplePendulum/}
{springActivity/}
{springProblemsI/}
{springProblemsII/}
{undCoeffHigherOrderEqs/}
{undeterminedCoeff/}
{undeterminedCoeff2/}
{unitStepFunction/}
{varParHigherOrder/}
{varParamNonHomLinSys/}
{variationOfParameters/}
}


\usepackage{tikz}
\usepackage{tkz-euclide}
\usepackage{tikz-3dplot}
\usepackage{tikz-cd}
\usetikzlibrary{shapes.geometric}
\usetikzlibrary{arrows}
\usetikzlibrary{decorations.pathmorphing,patterns}
\usetikzlibrary{backgrounds} % added by Felipe
% \usetkzobj{all}   % NOT ALLOWED IN RECENT TeX's ...
\pgfplotsset{compat=1.13} % prevents compile error.

\renewcommand{\vec}[1]{\mathbf{#1}}
\newcommand{\RR}{\mathbb{R}}
\newcommand{\dfn}{\textit}
\newcommand{\dotp}{\cdot}
\newcommand{\id}{\text{id}}
\newcommand\norm[1]{\left\lVert#1\right\rVert}
\newcommand{\dst}{\displaystyle}
 
\newtheorem{general}{Generalization}
\newtheorem{initprob}{Exploration Problem}

\tikzstyle geometryDiagrams=[ultra thick,color=blue!50!black]

\usepackage{mathtools}

\title{Euler's Method}


\begin{document}%\label{Module 5-1}

\begin{abstract}
Need Abstract
\end{abstract}

\maketitle

\section*{Euler's Method}

\subsection*{Introduction}
%from Mooculus
In science and mathematics, finding exact solutions to differential
equations is not always possible.  We have already seen that slope
fields give us a powerful way to understand the qualitative features
of solutions.  Sometimes we need a more precise quantitative
understanding, meaning we would like numerical approximations of the
solutions.

%from Trench
If an initial value problem
\begin{equation} \label{eq:3.1.1}
y'=f(x,y),\quad y(x_0)=y_0
 \end{equation}
can't be solved analytically,
 it's necessary to resort to
numerical methods to obtain useful  approximations to a solution.

The simplest numerical method for solving \eqref{eq:3.1.1} is
\href{http://www-history.mcs.st-and.ac.uk/Mathematicians/Euler.html}
\dfn{Euler's method}.
This method is so crude that it
is seldom used in practice;     however, its simplicity makes it useful
for illustrative purposes.

Euler's method is based on the assumption that the tangent line to the
integral curve of \eqref{eq:3.1.1} at $(x_i,y(x_i))$ approximates the
integral curve over the interval $[x_i,x_{i+1}]$.
 Since the slope of the integral curve of
\eqref{eq:3.1.1} at $(x_i,y(x_i))$ is $y'(x_i)=f(x_i,y(x_i))$, the
equation of the tangent line to the integral curve at $(x_i,y(x_i))$
is
\begin{equation} \label{eq:3.1.2}
y=y(x_i)+f(x_i,y(x_i))(x-x_i).
\end{equation}

Setting $x=x_{i+1}=x_i+h$ in \eqref{eq:3.1.2} yields
\begin{equation} \label{eq:3.1.3}
y_{i+1}=y(x_i)+hf(x_i,y(x_i))
\end{equation}
as an approximation to $y(x_{i+1})$. Since $y(x_0)=y_0$ is known, we
can use \eqref{eq:3.1.3} with $i=0$ to compute
$$
y_1=y_0+hf(x_0,y_0).
$$
However, setting $i=1$ in \eqref{eq:3.1.3} yields
$$
y_2=y(x_1)+hf(x_1,y(x_1)),
$$
which isn't  useful, since we don't know $y(x_1)$. Therefore
we replace $y(x_1)$ by its approximate value $y_1$ and redefine
$$
y_2=y_1+hf(x_1,y_1).
$$
Having computed $y_2$, we can  compute
$$
y_3=y_2+hf(x_2,y_2).
$$
In general, Euler's method starts with the known value $y(x_0)=y_0$
and computes $y_1, y_2, \ldots, y_n$ successively using the
 formula
\begin{equation} \label{eq:3.1.4}
y_{i+1}=y_i+hf(x_i,y_i),\quad 0\le i\le n-1.
\end{equation}

The next example illustrates the computational procedure
indicated in Euler's method.

%from Mooculus
\begin{example}\label{ex:eulerIntro1}
Consider the following differential equation
$$y' = x+y,\quad y(0)=1$$

Let us approximate the solution using just two subintervals
$$
\left[0,\frac{1}{2}\right]\qquad\text{and}\qquad \left[\frac{1}{2},1\right].
$$
Since $y'(0) = 1$, we know that
$$
y\left(\frac{1}{2}\right)\approx y(0)+\frac{1}{2}\cdot y'(0)= 1+ \frac{1}{2}\cdot 1 = \frac{3}{2}
$$
by \eqref{eq:3.1.3}.
% Commented out by Felipe
%\begin{image}
%  \begin{tikzpicture}
%    \begin{axis}[
%        domain=0:1, xmin =0,xmax=1.1,ymax=3,ymin=-.2,
%        width=6in,
%        height=3in,
%        %xticklabels={$1$,$1.5$,$2$, $2.5$, $3$},
%        %% ytick style={draw=none},
%        %% yticklabels={},
%        axis lines=center, xlabel=$t$, ylabel=$y$,
%            every axis y label/.style={at=(current axis.above origin),anchor=south},
%            every axis x label/.style={at=(current axis.right of origin),anchor=west},
%          ]
%          \draw[blue] (0,1)--(0.5, 1.5);
%      %\addplot [blue,very thick] plot coordinates {(0,1) (.5,1.5)};
%    \end{axis}
%\end{tikzpicture}
%\end{image}
But now, we have
$$
y'\left(\frac{1}{2}\right) \approx \frac{1}{2}+\frac{3}{2} = 2
$$
so
$$
y(1) \approx  y\left(\frac{1}{2}\right)+\frac{1}{2} \cdot 2= \frac{5}{2}
$$
% Commented out by Felipe
%\begin{image}
%  \begin{tikzpicture}
%    \begin{axis}[
%        domain=0:1, xmin =0,xmax=1.1,ymax=3,ymin=-.2,
%        width=6in,
%        height=3in,
%        %xticklabels={$1$,$1.5$,$2$, $2.5$, $3$},
%        %% ytick style={draw=none},
%        %% yticklabels={},
%        axis lines=center, xlabel=$t$, ylabel=$y$,
%            every axis y label/.style={at=(current axis.above origin),anchor=south},
%            every axis x label/.style={at=(current axis.right of origin),anchor=west},
%          ]
%        \draw[blue] (0,1)--(0.5, 1.5);
%        \draw[blue] (0.5,1.5)--(1, 2.5);
%      %\addplot [draw=blue,ultra thick] plot coordinates {(0,1) (.5,1.5)};
%      %\addplot [draw=blue,ultra thick] plot coordinates {(.5,1.5) (1,2.5)};
%    \end{axis}
%\end{tikzpicture}
%\end{image}
Plotting our approximation with the actual solution we find:
% Commented out by Felipe
%\begin{image}
%  \begin{tikzpicture}
%    \begin{axis}[
%        domain=0:1, xmin =0,xmax=1.1,ymax=3,ymin=-.2,
%        width=6in,
%        height=3in,
%        %xticklabels={$1$,$1.5$,$2$, $2.5$, $3$},
%        %% ytick style={draw=none},
%        %% yticklabels={},
%        axis lines=center, xlabel=$t$, ylabel=$y$,
%            every axis y label/.style={at=(current axis.above origin),anchor=south},
%            every axis x label/.style={at=(current axis.right of origin),anchor=west},
%          ]
%      %\addplot [draw=blue,very thick] plot coordinates {(0,1) (.5,1.5)};
%      %\addplot [draw=blue,very thick] plot coordinates {(.5,1.5) (1,2.5)};
%      \draw[blue] (0,1)--(0.5, 1.5);
%      \draw[blue] (0.5,1.5)--(1, 2.5);
%      \addplot [red, very thick] {2*e^x-x-1};
%    \end{axis}
%\end{tikzpicture}
%\end{image}


This approximation could be improved by using more subintervals.  Use the Sage Cell below to increase the number of subintervals to improve the approximation.

% Commented out by Felipe
%\begin{sageCell}% Anna
%@interact
%def _(endPoint=input_box(1, width=5),numberOfIntervals=input_box(1, width=5)):
%    a=var('a')
%    b=function('b')(a)
%    de = diff(b,a) ==  a+b
%    soln = desolve(de, [b, a], ics=[0,1])
%    P3=plot(soln(a), (a, 0, endPoint), color='red')
%
%    from sage.calculus.desolvers import eulers_method
%    x,y = PolynomialRing(QQ,2,"xy").gens()
%    pts=eulers_method(x+y,0,1,endPoint/numberOfIntervals,endPoint-(endPoint/numberOfIntervals),algorithm="none")
%
%    P1 = list_plot(pts)
%    P2 = line(pts)
%
%    (P1+P2+P3).show()
%\end{sageCell}

\end{example}

%Anna's example
\begin{example}\label{ex:eulerIntro2}
Use Euler's method to approximate $y$ on $[0,1]$ where $\d x = .25$ for
the differential equation
$$
y'=x-y
$$
where $y(0) = 1$.
\begin{explanation}

Fill out the following table for Euler's method using $4$
subintervals.
\[
\begin{array}{l|l}
   x & y \\ \hline
   0   & 1 \\
   0.25 & \answer{0.75} \\
   \answer{0.5} & \answer{5/8}  \\
   \answer{0.75} & \answer{19/32} \\
   1 & \frac{81}{128}
\end{array}
\]
Thus $y(1) \approx \answer{81/128}$.

The Sage Cell below illustrates your work graphically.  You can adjust the number of subintervals to see how the approximation is affected.  You can also change the the interval, just for fun.

% Commented out by Felipe
%\begin{sageCell}% Anna
%@interact
%def _(endPoint=input_box(1, width=5),numberOfIntervals=input_box(4, width=5)):
%    a=var('a')
%    b=function('b')(a)
%    de = diff(b,a) ==  a-b
%    soln = desolve(de, [b, a], ics=[0,1])
%    P3=plot(soln(a), (a, 0, endPoint), color='red')
%
%    from sage.calculus.desolvers import eulers_method
%    x,y = PolynomialRing(QQ,2,"xy").gens()
%    pts=eulers_method(x-y,0,1,endPoint/numberOfIntervals,endPoint-(endPoint/numberOfIntervals),algorithm="none")
%    
%    tableOfValues=eulers_method(x-y,0,1,endPoint/numberOfIntervals,endPoint,algorithm="table")
%    P1 = list_plot(pts)
%    P2 = line(pts)
%
%    (P1+P2+P3).show()
%\end{sageCell}

\end{explanation}
\end{example}


\subsection*{Sources of Error}

When using numerical methods we're interested in
computing approximate values of the solution of \eqref{eq:3.1.1} at
equally spaced points $x_0, x_1, \ldots, x_n=b$ in an interval
$[x_0,b]$.
 Thus,
$$
x_i=x_0+ih,\quad i=0,1, \dots,n,
$$
where
$$
h=\frac{b-x_0}{n}.
$$
We'll denote the approximate values of the solution at these points
by $y_0, y_1, \ldots, y_n$;   thus, $y_i$ is an approximation to
$y(x_i)$.
We'll call
$$
e_i=y(x_i)-y_i
$$
the \dfn{error at the $i$th step}. Because of the initial
condition
$y(x_0)=y_0$, we'll always have $e_0=0$. However, in general
$e_i\neq 0$ if $i>0$.

We encounter two sources of error
in applying a numerical method to solve an initial value problem:
\begin{itemize}
\item
The formulas defining the method are based on some sort of
approximation. Errors due to the inaccuracy of the approximation are
called \dfn{truncation errors}.
\item
Computers do arithmetic with a fixed number of digits, and therefore
make errors in evaluating the formulas defining the numerical methods.
Errors due to the computer's inability to do exact arithmetic are
called \dfn{roundoff errors}.
\end{itemize}

Since a careful analysis of roundoff error is beyond the scope of this
book, we'll consider only truncation errors.

\begin{example}\label{example:3.1.2}
Use Euler's method with step sizes $h=0.1$, $h=0.05$, and $h=0.025$ to
find approximate values of the solution of the initial value problem
$$
y'+2y=x^3e^{-2x},\quad y(0)=1
$$
at $x=0, 0.1, 0.2, 0.3, \ldots, 1.0$. Compare these approximate
values
with the values of the exact solution
\begin{equation} \label{eq:3.1.6}
y={e^{-2x}\over4}(x^4+4),
\end{equation}
which can be obtained by the method of Module 1-B. (Verify.)



\begin{explanation}
The table below shows the values of the exact solution
\eqref{eq:3.1.6} at the specified points, and the approximate values of
the solution at these points obtained by Euler's method with step
sizes $h=0.1$, $h=0.05$, and $h=0.025$. In examining this table, keep
in mind that the approximate values in the column corresponding to
$h=.05$ are actually the results of 20 steps with Euler's method. We
haven't listed the estimates of the solution obtained for
$x=0.05, 0.15, \dots $, since there's nothing to compare them with in
the column corresponding to $h=0.1$. Similarly, the approximate values
in the column corresponding to $h=0.025$ are actually the results of
$40$ steps with Euler's method.


$$
\begin{array}{|l|l|l|l|l|}
\hline
x & h=0.1 & h=0.05 & h=0.025 & \text{Exact}\\ \hline
0.0 & 1.000000000 & 1.000000000 & 1.000000000 & 1.000000000 \\
0.1 & 0.800000000 & 0.810005655 & 0.814518349 & 0.818751221 \\
0.2 & 0.640081873 & 0.656266437 & 0.663635953 & 0.670588174 \\
0.3 & 0.512601754 & 0.532290981 & 0.541339495 & 0.549922980 \\
0.4 & 0.411563195 & 0.432887056 & 0.442774766 & 0.452204669 \\
0.5 & 0.332126261 & 0.353785015 & 0.363915597 & 0.373627557 \\
0.6 & 0.270299502 & 0.291404256 & 0.301359885 & 0.310952904 \\
0.7 & 0.222745397 & 0.242707257 & 0.252202935 & 0.261398947 \\
0.8 & 0.186654593 & 0.205105754 & 0.213956311 & 0.222570721 \\
0.9 & 0.159660776 & 0.176396883 & 0.184492463 & 0.192412038 \\
1.0 & 0.139778910 & 0.154715925 & 0.162003293 & 0.169169104\\
\hline
\end{array}
$$

You can see that decreasing the step size
improves the accuracy of Euler's method. For example,
$$
y_{\text{exact}}(1)-y_{\text{approx}}(1)\approx
\left\{\begin{array}{l}
.0293\text{ with }h=0.1,\\
.0144\text{ with }h=0.05,\\
.0071\text{ with }h=0.025.
\end{array}\right.
$$
Based on this scanty evidence, you might guess that the error in
approximating the exact solution at a \textit{fixed value of} $x$ by
Euler's method is roughly halved when the step size is halved. You can
find more evidence to support this conjecture by examining
the table below which lists the approximate values of
$y_{\text{exact}}-y_{\text{approx}}$ at
$x=0.1, 0.2, \dots, 1.0$.

$$
\begin{array}{|l|l|l|l|}
\hline
x & h=0.1 & h=0.05 & h=0.025\\ \hline
0.1 & 0.0187 & 0.0087 & 0.0042\\
0.2 & 0.0305 & 0.0143 & 0.0069\\
0.3 & 0.0373 & 0.0176 & 0.0085\\
0.4 & 0.0406 & 0.0193 & 0.0094\\
0.5 & 0.0415 & 0.0198 & 0.0097\\
0.6 & 0.0406 & 0.0195 & 0.0095\\
0.7 & 0.0386 & 0.0186 & 0.0091\\
0.8 & 0.0359 & 0.0174 & 0.0086\\
0.9 & 0.0327 & 0.0160 & 0.0079\\
1.0 & 0.0293 & 0.0144 & 0.0071\\
\hline
\end{array}
$$

Use the Sage Cell below to explore the error visually by changing the size ($h$) of the subintervals.

% Commented out by Felipe
%\begin{sageCell} % Anna
%@interact
%def _(h=input_box(0.1, width=5)):
%    a=var('a')
%    b=function('b')(a)
%    de = diff(b,a) ==  (a^3)*e^(-2*a)-2*b
%    soln = desolve(de, [b, a], ics=[0,1])
%    P3=plot(soln(a), (a, 0, 1), color='red')
%
%    from sage.calculus.desolvers import eulers_method
%    x,y = PolynomialRing(QQ,2,"xy").gens()
%    pts=eulers_method((x^3)*e^(-2*x)-2*y,0,1,h,1-h,algorithm="none")
%
%    P1 = list_plot(pts)
%    P2 = line(pts)
%
%    (P1+P2+P3).show()
%\end{sageCell}

\href{https://docs.google.com/spreadsheets/d/1aE9Nh2VZpG2zJI70yeJpjNZDdxxaqFA8qwoZakjrK-M/edit?usp=sharing}{Spreadsheet}% Paul

\end{explanation}
\end{example}







\section*{Text Source}
\href{https://github.com/mooculus/calculus}{MOOCulus},
\href{https://github.com/mooculus/calculus/blob/a5d5bc54f341b1f3224dde40168188e81032ef65/differentialEquations/digInDifferentialEquations.tex#L1}{Numerical Methods} (CC-BY-NC-SA)


\end{document}