\documentclass{ximera}

%% You can put user macros here
%% However, you cannot make new environments

%\listfiles

% Get the 'old' hints/expandables, for use on ximera.osu.edu
\def\xmNotHintAsExpandable{true}
\def\xmNotExpandableAsAccordion{true}



%\graphicspath{{./}{firstExample/}{secondExample/}}
\graphicspath{{./}
{aboutDiffEq/}
{applicationsLeadingToDiffEq/}
{applicationsToCurves/}
{autonomousSecondOrder/}
{basicConcepts/}
{bernoulli/}
{constCoeffHomSysI/}
{constCoeffHomSysII/}
{constCoeffHomSysIII/}
{constantCoeffWithImpulses/}
{constantCoefficientHomogeneousEquations/}
{convolution/}
{coolingActivity/}
{directionFields/}
{drainingTank/}
{epidemicActivity/}
{eulersMethod/}
{exactEquations/}
{existUniqueNonlinear/}
{frobeniusI/}
{frobeniusII/}
{frobeniusIII/}
{global.css/}
{growthDecay/}
{heatingCoolingActivity/}
{higherOrderConstCoeff/}
{homogeneousLinearEquations/}
{homogeneousLinearSys/}
{improvedEuler/}
{integratingFactors/}
{interactExperiment/}
{introToLaplace/}
{introToSystems/}
{inverseLaplace/}
{ivpLaplace/}
{laplaceTable/}
{lawOfCooling/}
{linSysOfDiffEqs/}
{linearFirstOrderDiffEq/}
{linearHigherOrder/}
{mixingProblems/}
{motionUnderCentralForce/}
{nonHomogeneousLinear/}
{nonlinearToSeparable/}
{odesInSage/}
{piecewiseContForcingFn/}
{population/}
{reductionOfOrder/}
{regularSingularPts/}
{reviewOfPowerSeries/}
{rlcCircuit/}
{rungeKutta/}
{secondLawOfMotion/}
{separableEquations/}
{seriesSolNearOrdinaryPtI/}
{seriesSolNearOrdinaryPtII/}
{simplePendulum/}
{springActivity/}
{springProblemsI/}
{springProblemsII/}
{undCoeffHigherOrderEqs/}
{undeterminedCoeff/}
{undeterminedCoeff2/}
{unitStepFunction/}
{varParHigherOrder/}
{varParamNonHomLinSys/}
{variationOfParameters/}
}


\usepackage{tikz}
\usepackage{tkz-euclide}
\usepackage{tikz-3dplot}
\usepackage{tikz-cd}
\usetikzlibrary{shapes.geometric}
\usetikzlibrary{arrows}
\usetikzlibrary{decorations.pathmorphing,patterns}
\usetikzlibrary{backgrounds} % added by Felipe
% \usetkzobj{all}   % NOT ALLOWED IN RECENT TeX's ...
\pgfplotsset{compat=1.13} % prevents compile error.

\renewcommand{\vec}[1]{\mathbf{#1}}
\newcommand{\RR}{\mathbb{R}}
\providecommand{\dfn}{\textit}
\renewcommand{\dfn}{\textit}
\newcommand{\dotp}{\cdot}
\newcommand{\id}{\text{id}}
\newcommand\norm[1]{\left\lVert#1\right\rVert}
\newcommand{\dst}{\displaystyle}
 
\newtheorem{general}{Generalization}
\newtheorem{initprob}{Exploration Problem}

\tikzstyle geometryDiagrams=[ultra thick,color=blue!50!black]

\usepackage{mathtools}

\title{The Method of Frobenius III}%\label{Module 7-ADEF}


\begin{document}

\begin{abstract}
We conclude our study of the method of Frobenius for finding series solutions of linear second order differential equations, considering the case where the indicial equation has distinct real roots that differ by an integer.
\end{abstract}

\maketitle

\section*{The Method of Frobenius III}

In Trench \href{https://ximera.osu.edu/ode/main/frobeniusI/frobeniusI}{7.5} and \href{https://ximera.osu.edu/ode/main/frobeniusII/frobeniusII}{7.6} we discussed methods
for finding
Frobenius solutions of a homogeneous linear second order equation near
a regular singular point in the case where the indicial equation has a
repeated root or distinct real roots that don't differ by an integer.
In this section we consider the case where the indicial equation has
distinct real roots that differ by an integer. We'll limit our
discussion to equations that can be written as
\begin{equation} \label{eq:7.7.1}
x^2(\alpha_0+\alpha_1x)y''+x(\beta_0+\beta_1x)y'
+(\gamma_0+\gamma_1x)y=0
\end{equation}
or
$$
x^2(\alpha_0+\alpha_2x^2)y''+x(\beta_0+\beta_2x^2)y'
+(\gamma_0+\gamma_2x^2)y=0,
$$
where the roots of the indicial equation differ by a positive integer.

We begin with a theorem that provides a fundamental set of solutions
of equations of the form \eqref{eq:7.7.1}.


\begin{theorem}\label{thmtype:7.7.1}
Let
$$
Ly=
x^2(\alpha_0+\alpha_1x)y''+x(\beta_0+\beta_1x)y'
+(\gamma_0+\gamma_1x)y,
$$
where $\alpha_0\neq0,$ and  define
\begin{eqnarray*}
p_0(r)&=&\alpha_0r(r-1)+\beta_0r+\gamma_0,\\
p_1(r)&=&\alpha_1r(r-1)+\beta_1r+\gamma_1.
\end{eqnarray*}
Suppose   $r$ is a real number such that $p_0(n+r)$ is nonzero
for all positive integers $n,$ and  define
\begin{equation} \label{eq:7.7.2}
\begin{array}{rcl}
a_0(r)&=&1,\\
a_n(r)&=&-\frac{p_1(n+r-1)}{p_0(n+r)}a_{n-1}(r),\quad n\geq 1.
\end{array}
\end{equation}
Let $r_1$ and $r_2$ be the roots of the indicial equation
$p_0(r)=0,$ and suppose $r_1=r_2+k,$ where $k$ is a positive
integer$.$ Then
$$
y_1=x^{r_1}\sum_{n=0}^\infty  a_n(r_1)x^n
$$
is a Frobenius solution of $Ly=0$.
Moreover, if we define
\begin{equation} \label{eq:7.7.3}
\begin{array}{rcl}
a_0(r_2)&=&1,\\
a_n(r_2)&=&-\frac{p_1(n+r_2-1)}{p_0(n+r_2)}a_{n-1}(r_2),\quad
1\leq n\leq k-1,
\end{array}
\end{equation}
and
\begin{equation} \label{eq:7.7.4}
C=-\frac{p_1(r_1-1)}{k\alpha_0}a_{k-1}(r_2),
\end{equation}
then
\begin{equation} \label{eq:7.7.5}
y_2=x^{r_2}\sum_{n=0}^{k-1}a_n(r_2)x^n+C\left(y_1\ln
x+x^{r_1}\sum_{n=1}^\infty a_n'(r_1)x^n\right)
\end{equation}
is also a   solution of $Ly=0,$ and $\{y_1,y_2\}$
is a fundamental set of  solutions.
\end{theorem}

\begin{proof}  Theorem~\ref{thmtype:7.5.3} implies that $Ly_1=0$.
We'll now show that $Ly_2=0$.
Since $L$  is a linear operator, this is equivalent to
showing that
\begin{equation} \label{eq:7.7.6}
L\left(
x^{r_2}\sum_{n=0}^{k-1}a_n(r_2)x^n\right)+CL\left(y_1\ln
x+x^{r_1}\sum_{n=1}^\infty a_n'(r_1)x^n\right)=0.
\end{equation}
To verify this, we'll show that
\begin{equation} \label{eq:7.7.7}
L\left(x^{r_2}\sum_{n=0}^{k-1}
a_n(r_2)x^n\right)=p_1(r_1-1)a_{k-1}(r_2)x^{r_1}
\end{equation}
and
\begin{equation} \label{eq:7.7.8}
L\left(y_1\ln x+x^{r_1}\sum_{n=1}^\infty
a_n'(r_1)x^n\right)=k\alpha_0x^{r_1}.
\end{equation}
This will imply that $Ly_2=0$, since substituting \eqref{eq:7.7.7}
and \eqref{eq:7.7.8} into \eqref{eq:7.7.6} and using \eqref{eq:7.7.4}
yields
\begin{eqnarray*}
Ly_2&=&\left[p_1(r_1-1)a_{k-1}(r_2)+Ck\alpha_0\right]x^{r_1}\\
&=&\left[p_1(r_1-1)a_{k-1}(r_2)-p_1(r_1-1)a_{k-1}(r_2)\right]x^{r_1}=0.
\end{eqnarray*}


We'll prove \eqref{eq:7.7.8} first. From
Theorem~\ref{thmtype:7.6.1},
$$
L\left(y(x,r)\ln x+x^r\sum_{n=1}^\infty
a_n'(r)x^n\right)=p_0'(r)x^r+x^rp_0(r)\ln x.
$$
Setting $r=r_1$ and recalling that $p_0(r_1)=0$ and $y_1=y(x,r_1)$
yields
\begin{equation} \label{eq:7.7.9}
L\left(y_1\ln x+x^{r_1}\sum_{n=1}^\infty
a_n'(r_1)x^n\right)=p_0'(r_1)x^{r_1}.
\end{equation}
Since $r_1$ and $r_2$ are the roots of the indicial equation, the
indicial polynomial can be written as
$$
p_0(r)=\alpha_0(r-r_1)(r-r_2)=\alpha_0\left[r^2-(r_1+r_2)r+r_1r_2\right].
$$
Differentiating this yields
$$
p_0'(r)=\alpha_0(2r-r_1-r_2).
$$
Therefore $p_0'(r_1)=\alpha_0(r_1-r_2)=k\alpha_0$,
so \eqref{eq:7.7.9} implies \eqref{eq:7.7.8}.

Before proving \eqref{eq:7.7.7}, we first note $a_n(r_2)$ is well
defined by \eqref{eq:7.7.3} for $1\leq n\leq k-1$, since $p_0(n+r_2)\neq 0$
for these values of $n$. However, we can't define $a_n(r_2)$ for
$n\geq k$ with \eqref{eq:7.7.3}, since $p_0(k+r_2)=p_0(r_1)=0$.
For convenience, we define  $a_n(r_2)=0$ for $n\geq k$. Then, from
Theorem~\ref{thmtype:7.5.1},
\begin{equation} \label{eq:7.7.10}
L\left(x^{r_2}\sum_{n=0}^{k-1} a_n(r_2)x^n\right)=
L\left(x^{r_2}\sum_{n=0}^\infty a_n(r_2)x^n\right)=
x^{r_2}\sum_{n=0}^\infty b_nx^n,
\end{equation}
where $b_0=p_0(r_2)=0$  and
$$
b_n=p_0(n+r_2)a_n(r_2)+p_1(n+r_2-1)a_{n-1}(r_2),\quad n\geq 1.
$$
If $1\leq n\leq k-1$, then \eqref{eq:7.7.3} implies that $b_n=0$.
If $n\geq k+1$, then $b_n=0$  because $a_{n-1}(r_2)=a_n(r_2)=0$.
Therefore \eqref{eq:7.7.10} reduces to
$$
L\left(x^{r_2}\sum_{n=0}^{k-1} a_n(r_2)x^n\right)=
\left[p_0(k+r_2)a_k(r_2)+p_1(k+r_2-1)a_{k-1}(r_2)
\right]x^{k+r_2}.
$$
Since $a_k(r_2)=0$ and $k+r_2=r_1$, this implies \eqref{eq:7.7.7}.

We leave the proof that $\{y_1,y_2\}$ is a fundamental set as an
exercise. %(Exercise~\ref{exer:7.7.41}).
\end{proof}

\begin{example}\label{example:7.7.1}
Find a fundamental set of Frobenius  solutions of
$$
2x^2(2+x)y''-x(4-7x)y'-(5-3x)y=0.
$$
Give explicit  formulas for the coefficients in the solutions.

\begin{explanation}
For  the given equation, the polynomials defined in
Theorem~\ref{thmtype:7.7.1} are
$$
\begin{array}{ccccc}
p_0(r)&=&4r(r-1)-4r-5&=&(2r+1)(2r-5),\\
p_1(r)&=&2r(r-1)+7r+3&=&(r+1)(2r+3).
\end{array}
$$
The roots of the indicial equation are $r_1=5/2$ and $r_2=-1/2$,
so $k=r_1-r_2=3$. Therefore Theorem~\ref{thmtype:7.7.1} implies that
\begin{equation} \label{eq:7.7.11}
y_1=x^{5/2}\sum_{n=0}^\infty a_n(5/2)x^n
\end{equation}
and
\begin{equation} \label{eq:7.7.12}
y_2=x^{-1/2}\sum_{n=0}^2a_n(-1/2)+C\left(y_1\ln
x+x^{5/2}\sum_{n=1}^\infty a_n'(5/2)x^n\right)
\end{equation}
(with $C$ as in \eqref{eq:7.7.4})
form a fundamental set of solutions of $Ly=0$. The recurrence
formula \eqref{eq:7.7.2} is
\begin{equation} \label{eq:7.7.13}
\begin{array}{ccl}
a_0(r)&=&1,\\
a_n(r)&=&-\frac{p_1(n+r-1)}{p_0(n+r)}a_{n-1}(r)\\
&=&-\frac{(n+r)(2n+2r+1)}{(2n+2r+1)(2n+2r-5)}a_{n-1}(r),\\
&=&-\frac{n+r}{2n+2r-5}a_{n-1}(r),\,n\geq 1,
\end{array}
\end{equation}
which implies that
\begin{equation} \label{eq:7.7.14}
a_n(r)=(-1)^n\prod_{j=1}^n\frac{j+r}{2j+2r-5},\,n\geq 0.
\end{equation}
Therefore
\begin{equation} \label{eq:7.7.15}
a_n(5/2)=\frac{(-1)^n\prod_{j=1}^n(2j+5)}{4^nn!}.
\end{equation}
Substituting this into \eqref{eq:7.7.11} yields
$$
y_1=x^{5/2}\sum_{n=0}^\infty\frac{(-1)^n\prod_{j=1}^n(2j+5)}{4^nn!}x^n.
$$

To compute the coefficients $a_0(-1/2),a_1(-1/2)$ and $a_2(-1/2)$ in
$y_2$,
we set $r=-1/2$ in \eqref{eq:7.7.13} and apply the resulting recurrence
formula for $n=1$, $2$;   thus,
\begin{eqnarray*}
a_0(-1/2)&=&1,\\
a_n(-1/2)&=&-\frac{2n-1}{4(n-3)}a_{n-1}(-1/2),\,n=1,2.
\end{eqnarray*}
The last formula  yields
$$
a_1(-1/2)=1/8 \quad\mbox{and}\quad a_2(-1/2)=3/32.
$$
Substituting $r_1=5/2,r_2=-1/2,k=3$, and $\alpha_0=4$ into
 \eqref{eq:7.7.4} yields $C=-15/128$.  Therefore, from
\eqref{eq:7.7.12},
\begin{equation} \label{eq:7.7.16}
y_2=x^{-1/2}\left(1+\frac{1}{8}x+\frac{3}{32}x^2\right)
-\frac{15}{128}
\left(y_1\ln x+x^{5/2}\sum_{n=1}^\infty a_n'(5/2)x^n\right).
\end{equation}

We use logarithmic differentiation to obtain
 obtain $a'_n(r)$. From
\eqref{eq:7.7.14},
$$
|a_n(r)|=\prod_{j=1}^n{|j+r|\over|2j+2r-5|},\,n\geq 1.
$$
Therefore
$$
\ln |a_n(r)|=\sum^n_{j=1} \left(\ln |j+r|-\ln|2j+2r-5|\right).
$$
Differentiating  with respect to $r$ yields
$$
\frac{a'_n(r)}{a_n(r)}=\sum^n_{j=1} \left(\frac{1}{j+r}-\frac{2}{2j+2r-5}\right).
$$
Therefore
$$
a'_n(r)=a_n(r) \sum^n_{j=1} \left(\frac{1}{j+r}-\frac{2}{2j+2r-5}\right).
$$
Setting $r=5/2$ here and recalling \eqref{eq:7.7.15} yields
\begin{equation} \label{eq:7.7.17}
a_n'(5/2)=\frac{(-1)^n\prod_{j=1}^n(2j+5)}{4^nn!}\sum_{j=1}^n
\left(\frac{1}{j+5/2}-\frac{1}{j}\right).
\end{equation}
Since
$$
\frac{1}{j+5/2}-\frac{1}{j}=-\frac{5}{j(2j+5)},
$$
we can rewrite \eqref{eq:7.7.17} as
$$
a_n'(5/2)=-5\frac{(-1)^n\prod_{j=1}^n(2j+5)}{4^nn!}
\left(\sum_{j=1}^n\frac{1}{j(2j+5)}\right).
$$

Substituting this into \eqref{eq:7.7.16} yields
\begin{eqnarray*}
y_2&=&x^{-1/2}\left(1+\frac{1}{8}x+\frac{3}{32}x^2\right)-\frac{15}{128}
y_1\ln x\\
&&\,+\frac{75}{128}
x^{5/2}\sum_{n=1}^\infty
\frac{(-1)^n\prod_{j=1}^n(2j+5)}{4^nn!}
\left(\sum_{j=1}^n\frac{1}{j(2j+5)}\right)x^n.
\end{eqnarray*}

\end{explanation}
\end{example}

If $C=0$ in \eqref{eq:7.7.4}, there's no need to compute
$$
y_1\ln x+x^{r_1}\sum_{n=1}^\infty a_n'(r_1)x^n
$$
in the formula \eqref{eq:7.7.5} for $y_2$. Therefore it's best to
compute $C$  before computing $\{a_n'(r_1)\}_{n=1}^\infty$.
This is illustrated  in the next example. 
%(See also
%Exercises~\ref{exer:7.7.44}  and \ref{exer:7.7.45}.)



\begin{example}\label{example:7.7.2}
Find a fundamental set of Frobenius  solutions of
$$
x^2(1-2x)y''+x(8-9x)y'+(6-3x)y=0.
$$
Give explicit  formulas for the coefficients in the solutions.

\begin{explanation}
For  the given equation, the polynomials defined in
Theorem~\ref{thmtype:7.7.1} are
$$
\begin{array}{ccccc}
p_0(r)&=&r(r-1)+8r+6&=&(r+1)(r+6)\\
p_1(r)&=&-2r(r-1)-9r-3&=&-(r+3)(2r+1).
\end{array}
$$
The roots of the indicial equation  are $r_1=-1$ and $r_2=-6$,
so $k=r_1-r_2=5$. Therefore Theorem~\ref{thmtype:7.7.1} implies that
\begin{equation} \label{eq:7.7.18}
y_1=x^{-1}\sum_{n=0}^\infty a_n(-1)x^n
\end{equation}
and
\begin{equation} \label{eq:7.7.19}
y_2=x^{-6}\sum_{n=0}^4a_n(-6)+C\left(y_1\ln
x+x^{-1}\sum_{n=1}^\infty a_n'(-1)x^n\right)
\end{equation}
(with $C$ as in \eqref{eq:7.7.4})
form a fundamental set of solutions of $Ly=0$. The
recurrence formula \eqref{eq:7.7.2} is
\begin{equation} \label{eq:7.7.20}
\begin{array}{ccl}
a_0(r)&=&1,\\
a_n(r)&=&-\frac{p_1(n+r-1)}{p_0(n+r)}a_{n-1}(r)\\
&=&\frac{(n+r+2)(2n+2r-1)}{(n+r+1)(n+r+6)}a_{n-1}(r),\,n\geq 1,
\end{array}
\end{equation}
which implies that
\begin{equation} \label{eq:7.7.21}
\begin{array}{ccl}
a_n(r)&=&\prod_{j=1}^n\frac{(j+r+2)(2j+2r-1)}{(j+r+1)(j+r+6)}\\
&=&\left(\prod_{j=1}^n\frac{j+r+2}{j+r+1}\right)
\left(\prod_{j=1}^n\frac{2j+2r-1}{j+r+6}\right).
\end{array}
\end{equation}
Since
$$
\prod_{j=1}^n\frac{j+r+2}{j+r+1}=\frac{(r+3)(r+4)\cdots(n+r+2)}{(r+2)(r+3)\cdots(n+r+1)}=\frac{n+r+2}{r+2}
$$
because of cancellations, \eqref{eq:7.7.21}  simplifies to
$$
a_n(r)=\frac{n+r+2}{r+2}\prod_{j=1}^n\frac{2j+2r-1}{j+r+6}.
$$
Therefore
$$
a_n(-1)=(n+1)\prod_{j=1}^n\frac{2j-3}{j+5}.
$$
Substituting this into \eqref{eq:7.7.18} yields
$$
y_1=x^{-1}\sum_{n=0}^\infty (n+1)\left(\prod_{j=1}^n\frac{2j-3}{j+5}\right) x^n.
$$

To compute the coefficients
$a_0(-6),\dots,a_4(-6)$ in $y_2$, we
set $r=-6$ in \eqref{eq:7.7.20} and apply the resulting recurrence
formula for $n=1$, $2$, $3$, $4$;   thus,
\begin{eqnarray*}
a_0(-6)&=&1,\\
a_n(-6)&=&\frac{(n-4)(2n-13)}{n(n-5)}a_{n-1}(-6),\,n=1,2,3,4.
\end{eqnarray*}
The last formula yields
$$
a_1(-6)=-\frac{33}{4},\,a_2(-6)=\frac{99}{4},\,a_3(-6)=-\frac{231}{8},\,a_4(-6)=0.
$$
Since $a_4(-6)=0$, \eqref{eq:7.7.4} implies that the constant $C$
in \eqref{eq:7.7.19} is zero. Therefore \eqref{eq:7.7.19} reduces
to
$$
y_2=x^{-6}\left(1-\frac{33}{4}x+\frac{99}{4}x^2-\frac{231}{8}x^3\right).
$$
\end{explanation}
\end{example}

We now consider   equations  of the form
$$
x^2(\alpha_0+\alpha_2x^2)y''+x(\beta_0+\beta_2x^2)y'
+(\gamma_0+\gamma_2x^2)y=0,
$$
where the roots of the indicial equation are real and differ by an
even integer. 
%The case where the roots are real and differ by an odd
%integer can be handled by the method discussed in
%\ref{exer:7.5.56}.
The  proof of the next theorem is similar to the proof of
Theorem~\ref{thmtype:7.7.1} %(Exercise~\ref{exer:7.7.43}).

\begin{theorem}\label{thmtype:7.7.2}
Let
$$
Ly=
x^2(\alpha_0+\alpha_2x^2)y''+x(\beta_0+\beta_2x^2)y'
+(\gamma_0+\gamma_2x^2)y,
$$
where $\alpha_0\neq0,$ and  define
\begin{eqnarray*}
p_0(r)&=&\alpha_0r(r-1)+\beta_0r+\gamma_0,\\
p_2(r)&=&\alpha_2r(r-1)+\beta_2r+\gamma_2.
\end{eqnarray*}
Suppose    $r$ is a real number such that $p_0(2m+r)$ is nonzero
for all positive integers $m$, and  define
\begin{equation} \label{eq:7.7.22}
\begin{array}{rcl}
a_0(r)&=&1,\\
a_{2m}(r)&=&-\frac{p_2(2m+r-2)}{p_0(2m+r)}a_{2m-2}(r),\quad m\geq 1.
\end{array}
\end{equation}
Let $r_1$ and $r_2$ be the roots of the indicial equation
$p_0(r)=0,$ and suppose   $r_1=r_2+2k,$ where $k$ is a positive
integer.  Then
$$
y_1=x^{r_1}\sum_{m=0}^\infty  a_{2m}(r_1)x^{2m}
$$
is a Frobenius solution of $Ly=0$.
Moreover, if we define
\begin{eqnarray*}
a_0(r_2)&=&1,\\
a_{2m}(r_2)&=&-\frac{p_2(2m+r_2-2)}{p_0(2m+r_2)}a_{2m-2}(r_2),\quad
1\leq m\leq k-1
\end{eqnarray*}
and
\begin{equation} \label{eq:7.7.23}
C=-\frac{p_2(r_1-2)}{2k\alpha_0}a_{2k-2}(r_2),
\end{equation}
then
\begin{equation} \label{eq:7.7.24}
y_2=x^{r_2}\sum_{m=0}^{k-1}a_{2m}(r_2)x^{2m}+C\left(y_1\ln
x+x^{r_1}\sum_{m=1}^\infty a_{2m}'(r_1)x^{2m}\right)
\end{equation}
is also a   solution of $Ly=0,$ and $\{y_1,y_2\}$
is a fundamental set of  solutions.
\end{theorem}


\begin{example}\label{example:7.7.3}
Find a fundamental set of Frobenius  solutions of
$$
x^2(1+x^2)y''+x(3+10x^2)y'-(15-14x^2)y=0.
$$
Give explicit  formulas for the coefficients in the solutions.

\begin{explanation}
For  the given equation, the polynomials defined in
Theorem~\ref{thmtype:7.7.2} are
$$
\begin{array}{ccccc}
p_0(r)&=&r(r-1)+3r-15&=&(r-3)(r+5)\\
p_2(r)&=&r(r-1)+10r+14&=&(r+2)(r+7).
\end{array}
$$
The roots of the indicial equation are $r_1=3$ and $r_2=-5$,
so $k=(r_1-r_2)/2=4$. Therefore Theorem~\ref{thmtype:7.7.2} implies that
\begin{equation} \label{eq:7.7.25}
y_1=x^3\sum_{m=0}^\infty a_{2m}(3)x^{2m}
\end{equation}
and
$$
y_2=x^{-5}\sum_{m=0}^3 a_{2m}(-5)x^{2m}+C
\left(y_1\ln x+x^3\sum_{m=1}^\infty a_{2m}'(3)x^{2m}\right)
$$
(with $C$ as in \eqref{eq:7.7.23})
form a fundamental set of solutions of $Ly=0$. The recurrence formula
\eqref{eq:7.7.22} is
\begin{equation} \label{eq:7.7.26}
\begin{array}{ccl}
a_0(r)&=&1,\\
a_{2m}(r)&=&-\frac{p_2(2m+r-2)}{p_0(2m+r)}a_{2m-2}(r)\\
&=&-\frac{(2m+r)(2m+r+5)}{(2m+r-3)(2m+r+5)}a_{2m-2}(r)\\
&=&-\frac{2m+r}{2m+r-3}a_{2m-2}(r),\,m\geq 1,
\end{array}
\end{equation}
which implies that
\begin{equation} \label{eq:7.7.27}
a_{2m}(r)=(-1)^m\prod_{j=1}^m\frac{2j+r}{2j+r-3},\,m\geq 0.
\end{equation}
Therefore
\begin{equation} \label{eq:7.7.28}
a_{2m}(3)=\frac{(-1)^m\prod_{j=1}^m(2j+3)}{2^mm!}.
\end{equation}
Substituting this into \eqref{eq:7.7.25} yields
$$
y_1=x^3\sum_{m=0}^\infty
\frac{(-1)^m\prod_{j=1}^m(2j+3)}{2^mm!}x^{2m}.
$$

To compute the coefficients $a_2(-5)$, $a_4(-5)$, and $a_6(-5)$ in
$y_2$, we
set $r=-5$ in \eqref{eq:7.7.26} and apply the resulting recurrence
formula for $m=1$, $2$,  $3$;   thus,
$$
a_{2m}(-5)=-\frac{2m-5}{2(m-4)}a_{2m-2}(-5),\,m=1,2,3.
$$

This yields
$$
a_2(-5)=-\frac{1}{2},\,a_4(-5)=\frac{1}{8},\,a_6(-5)=\frac{1}{16}.
$$
Substituting $r_1=3$, $r_2=-5$, $k=4$, and $\alpha_0=1$   into
\eqref{eq:7.7.23}
yields $C=-3/16$.   Therefore, from \eqref{eq:7.7.24},
\begin{equation} \label{eq:7.7.29}
y_2=x^{-5} \left(1-\frac{1}{2}x^2+\frac{1}{8}x^4+\frac{1}{16}x^6\right)
-\frac{3}{16}
\left(y_1\ln x+x^3\sum_{m=1}^\infty a_{2m}'(3)x^{2m}\right).
\end{equation}

To obtain $a_{2m}'(r)$  we use logarithmic differentiation. From
\eqref{eq:7.7.27},
$$
|a_{2m}(r)|=\prod_{j=1}^m\frac{|2j+r|}{|2j+r-3|},\,m\geq 1.
$$
Therefore
$$
\ln |a_{2m}(r)|=\sum^n_{j=1} \left(\ln |2j+r|-\ln|2j+r-3|\right).
$$
Differentiating  with respect to $r$ yields
$$
\frac{a'_{2m}(r)}{a_{2m}(r)}=\sum^m_{j=1} \left(\frac{1}{2j+r}-\frac{1}{2j+r-3}\right).
$$
Therefore
$$
a'_{2m}(r)=a_{2m}(r) \sum^n_{j=1} \left(\frac{1}{2j+r}-\frac{1}{2j+r-3}\right).
$$
Setting $r=3$ here and recalling \eqref{eq:7.7.28} yields
\begin{equation} \label{eq:7.7.30}
a_{2m}'(3)=\frac{(-1)^m\prod_{j=1}^m(2j+3)}{2^mm!}\sum_{j=1}^m
\left(\frac{1}{2j+3}-\frac{1}{2j}\right).
\end{equation}
Since
$$
\frac{1}{2j+3}-\frac{1}{2j}=-\frac{3}{2j(2j+3)},
$$
we can rewrite \eqref{eq:7.7.30} as
$$
a_{2m}'(3)=-\frac{3}{2}\frac{(-1)^n\prod_{j=1}^m(2j+3)}{2^mm!}
\left(\sum_{j=1}^n\frac{1}{j(2j+3)}\right).
$$
Substituting this into \eqref{eq:7.7.29} yields
\begin{eqnarray*}
y_2&=&x^{-5}
\left(1-\frac{1}{2}x^2+\frac{1}{8}x^4+\frac{1}{16}x^6\right)
-\frac{3}{16}y_1\ln x \\
&&\, +\frac{9}{32}
x^3\sum_{m=1}^\infty
\frac{(-1)^m\prod_{j=1}^m(2j+3)}{2^mm!}\left(\sum_{j=1}^m\frac{1}{j(2j+3)}\right) x^{2m}.
\end{eqnarray*}
\end{explanation}
\end{example}


\begin{example}\label{example:7.7.4}
Find a fundamental set of Frobenius  solutions of
$$
x^2(1-2x^2)y''+x(7-13x^2)y'-14x^2y=0.
$$
Give explicit formulas for the coefficients in the solutions.

\begin{explanation}
For  the given equation, the polynomials defined in
Theorem~\ref{thmtype:7.7.2} are
$$
\begin{array}{ccccc}
p_0(r)&=&r(r-1)+7r&=&r(r+6),\\
p_2(r)&=&-2r(r-1)-13r-14&=&-(r+2)(2r+7).
\end{array}
$$
The roots of the indicial equation are $r_1=0$ and $r_2=-6$,
so $k=(r_1-r_2)/2=3$. Therefore Theorem~\ref{thmtype:7.7.2} implies that
\begin{equation} \label{eq:7.7.31}
y_1=\sum_{m=0}^\infty a_{2m}(0)x^{2m},
\end{equation}
and
\begin{equation} \label{eq:7.7.32}
y_2=x^{-6}\sum_{m=0}^2a_{2m}(-6)x^{2m}+C\left(y_1\ln
x+\sum_{m=1}^\infty a_{2m}'(0)x^{2m}\right)
\end{equation}
(with $C$ as in \eqref{eq:7.7.23})
form a fundamental set of
solutions of $Ly=0$. The recurrence formulas \eqref{eq:7.7.22} are
\begin{equation} \label{eq:7.7.33}
\begin{array}{ccl}
a_0(r)&=&1,\\
a_{2m}(r)&=&-\frac{p_2(2m+r-2)}{p_0(2m+r)}a_{2m-2}(r)\\
&=&\frac{(2m+r)(4m+2r+3)}{(2m+r)(2m+r+6)}a_{2m-2}(r)\\
&=&\frac{4m+2r+3}{2m+r+6}a_{2m-2}(r),\,m\geq 1,
\end{array}
\end{equation}
which implies that
$$
a_{2m}(r)=\prod_{j=1}^m\frac{4j+2r+3}{2j+r+6}.
$$
Setting $r=0$  yields
$$
a_{2m}(0)=6\frac{\prod_{j=1}^m(4j+3)}{2^m(m+3)!}.
$$
Substituting this into \eqref{eq:7.7.31} yields
$$
y_1=6\sum_{m=0}^\infty \frac{\prod_{j=1}^m(4j+3)}{2^m(m+3)!}x^{2m}.
$$

To compute the coefficients $a_0(-6)$, $a_2(-6)$, and $a_4(-6)$ in
$y_2$, we
set $r=-6$ in \eqref{eq:7.7.33} and apply the resulting recurrence
formula for $m=1$, $2$;   thus,
\begin{eqnarray*}
a_0(-6)&=&1,\\
a_{2m}(-6)&=&\frac{4m-9}{2m}a_{2m-2}(-6),\,m=1,2.
\end{eqnarray*}
The last formula yields
$$
a_2(-6)=-\frac{5}{2}\quad\mbox{and}\quad a_4(-6)=\frac{5}{8}.
$$
Since $p_2(-2)=0$, the constant $C$  in \eqref{eq:7.7.23} is zero.
Therefore \eqref{eq:7.7.32} reduces to
$$
y_2=x^{-6}\left(1-\frac{5}{2}x^2+\frac{5}{8}x^4\right).
$$
\end{explanation}
\end{example}

\section*{Text Source}
Trench, William F., "Elementary Differential Equations" (2013). Faculty Authored and Edited Books \& CDs. 8. (CC-BY-NC-SA)

\href{https://digitalcommons.trinity.edu/mono/8/}{https://digitalcommons.trinity.edu/mono/8/}


\end{document}